\section{Optionen \& Aktionen}

% Siehe auch: https://sopra.informatik.uni-freiburg.de/soprawiki/Game_Mechanic

% Dieser Abschnitt beinhaltet die Aktionen, die Spieler oder KI vornehmen
% können, um den Zustand des Spiels zu verändern (zB das Bauen von Einheiten
% oder das Abbauen von Ressourcen). Je klarer diese Aktionen formuliert sind,
% desto leichter fällt einem die Umsetzung der Aktionen bei der Programmierung
% des Spiels.
%
% Wichtig sind auch die Einstellungen, die der Spieler am Spiel vornehmen kann,
% um das Spielverhalten zu verändern (zB.  Schwierigkeitsgrad ändern).
%
% Das Ziel des Spiels sollte schließlich anhand der beschriebenen Aktionen
% erklärt werden.
%
% Die Auflistung der Optionen und Aktionen erfolgt tabellarisch und ist in Form
% und Inhalt an Use Cases (http://de.wikipedia.org/wiki/Use_case) angelehnt.

\missingSection{Optionen \& Aktionen}

{\small
\begin{longtabu}{X[0.6L]X[0.4L]X[L]X[0.9L]X[0.9L]}
  \toprule
  \rowfont[c]{\itshape}
    \multirow{1}{=}{\centering ID/Name}              &
    \multirow{1}{=}{\centering Akteure}              &
    \multirow{1}{=}{\centering Ereignis"-fluss}      &
    \multirow{1}{=}{Anfangs"-bedingung}   &
                   {Abschluss"-bedingung} \\
  \midrule\endhead

  % Example line taken from the SOPRA wiki.
  ID01: Figur durch Klick bewegen
    & Spieler
    & \dots
    & Der Spieler muss eine oder mehr kontrollierbare, auswählbare Spielfiguren
      ausgewählt haben.
    & Die Spielfiguren befinden sich am Zielpunkt, \textbf{oder} die
      Spielfiguren befinden an einem begehbaren Punkt in der Welt, der möglichst
      nah am Ziekpunkt liegt.
  \\

  \bottomrule
\end{longtabu}}
