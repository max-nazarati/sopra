\section{Spielstruktur}

% Dieser Abschnitt erklärt den Ablauf des Spiels. Das heißt, hier wird
% beschreiben, was geschieht, sobald der Spieler ein neues Spiel beginnt und
% wie sich das Spiel von dort aus entwickelt, bis es gewonnen oder verloren
% ist. Eine Beschreibung der unterschiedlichen Spielphasen ist hier essentiell.
%
% Eine mögliche Einteilung der Spielphasen von Schach ist zum Beispiel:
% Early-Game (Eröffnung), Mid-Game (strategische Positionen festigen),
% Late-Game (wenn nur noch wenige Figuren auf dem Brett sind).
%
% Eine weitere wichtige Information in diesem Abschnitt ist, welche Modi das
% Spiel hat (zum Beispiel Missionen und wie sie sich unterscheiden vs.
% Endlosmodus und wie dieser während des Spiels verändert wird).
%
% Außerdem soll die Dynamik des Spiels beschrieben werden (Beispiel: statisch,
% d.h. wenig Veränderungen an der Spielwelt und den Mechaniken vs. actionreich
% und dynamisch).
%
% Auch mögliche Taktiken und Strategien im Spiel können hier beschrieben
% werden.

Das Spiel ist ein dynamisches Spiel. 

\subsection{Kaufoptionen}
\label{subsec:kaufoptionen}
    Alle Angriffs- und Verteidigungseinheiten, deren Verbesserungen und deren
    Upgrades können zu jedem Zeitpunkt während des Spiels gekauft werden, 
    solange genug Gold vorhanden ist. Unterschiede bestehen beim 
    Aktivierungszeitpunkt.

	\begin{enumerate}
		\item Verbesserungen: Die Veränderung tritt sofort in Kraft.
		\item Upgrades: Die Veränderung tritt sofort in Kraft.
    		\item Kaufen neuer Einheiten
		\begin{enumerate}
			\item Türme: Sobald der gewählte Platz leer ist wird
				gebaut.
			\item Angriffseinheiten: Käufe bildet die Angreifer 
				für die nächste Welle.
    		\end{enumerate}
	\end{enumerate}

\subsection{Spielablauf}
Beim Starten eines neuen Spiels erscheint ein Feld mit der
Hintergrundgeschichte. Hier wird kurz in die Idee hinter der Spielwelt
eingeführt. //
Wenn das Feld geschlossen landet man bei der ersten Spielwelt. Jetzt hat der 
Spieler Zeit die erste Welle vorzubereiten, sprich seine Türme aufzustellen 
und seine Angriffseinheiten zu kaufen. Dafür hat der Spieler 50 Einheiten Gold
zur Verfügung. //
Nun kann die erste Welle gestartet werden.
Nun laufen die Angriffseinheiten in kurzen Abständen hintereinander von der
eigenen Basis los, auf die gegnerischen Türme zu. Die Reihenfolge entspricht
der Kaufreihenfolge. Zusätzlich tauchen die drei kontrollierbaren,
kollidierenden und bewegbaren Einheiten (Helden) am Ausgang der Basis auf. Diese
können von nun an mit der Maus gesteuert werden. \\
Sobald das Spiel gestartet wird verdient der Spieler über Zeit Gold. Dieses
kann auch sofort wieder investiert werden. Für jedes Goldstück besteht die
Wahl zwischen Angriff und Verteidigung(\ref{subsec:kaufoptionen}). \\

Die nächste Welle wird gestartet sobald einer der beiden Spieler keine 
Angriffseinheiten, mit Ausnahme der Helden, mehr auf dem Spielfeld besitzt
und beide Basen noch Leben besitzen. Die verbliebenen Einheiten des zweiten 
Spielers attakieren weiter bis sie ebenfalls tot sind. \\
Sobald die gesamte Angriffseinheit, mit Ausnahme der Helden, tot ist oder die 
gegnerische Basis erreicht hat, bekommt dieser Spieler einen Erfahrungspunkt. 
Außerdem bekommt der Spieler mit der höheren Anzahl an Angreifern die 
Differenz der Anzahl der Angreifer der beiden Spieler in Gold gut geschrieben.
Die Höhe dieses Betrags ist jedoch pro Welle limitiert und das Limit steigt 
mit zunehmender Anzahl an Wellen an.\\

Dieser Prozess wiederholt sich solange bis einer der beiden Basen zerstört 
wurde. Der Spieler, der die Basis zerstört hat, gewinnt das Spiel. 
Jetzt erscheint ein Gewonnen oder Verloren Bildschirm vom dem in das Hauptmenü
zurück gekehrt werden kann.
