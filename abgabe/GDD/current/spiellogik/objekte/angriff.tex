\subsection{Angriffseinheiten}

\begin{description}
  \item[Truppen]\todo[noline]{Benennung überdenken}
    kosten relativ wenig, lassen sich jedoch nicht weiter kontrollieren. Diese
    Einheiten verfolgen das Ziel, möglichst schnell zum gegenerischen Lager zu
    gelangen um dort Schaden zu verursachen.

  \item[Helden] kosten mehr als Truppen, diese Einheiten lassen sich jedoch vom
    Spieler kontrollieren und so strategisch einsetzen und außerhalb der
    Reichweite von Verteidigungsgebäuden positionieren; zusätzlich besitzen sie
    Fähigkeiten, die der Spieler einsetzen kann.

\end{description}

Tabelle~\ref{tab:attack-unit-props} beschreibt die Eigenschaften die
Angriffseinheiten haben, in Tabelle~\ref{tab:attack-units} sind alle Truppen
mit ihren Eigenschaften aufgelistet und Tabelle~\ref{tab:attack-heroes} enthält
alle Helden.

\begin{table}[htbp]
  \caption{Eigenschaften von Angriffseinheiten}
  \label{tab:attack-unit-props}
  \small
  \begin{longtabu}{rlX}
    \toprule\rowfont{\itshape}
    & Eigenschaft & Beschreibung \\
    \midrule

    B  & Beschreibung
       & Eine allgemeine Beschreibung dieser Einheit und Vergleich zu anderen
         Einheiten. \\
    F  & Fähigkeit
       & Nur Helden haben eine Fähigkeit, diese kann vom Spieler aktiviert
         werden (\refid{A:hero-ability}). \\
    K  & Kosten
       & Die Menge an Gold die aufgewendet werden muss, um eine dieser
         Einheiten zu kaufen (\refid{A:buy-attack}). \\
    LP & Lebenspunkte
       & Die Zahl der Lebenspunkte einer Einheit: Angriffe von
         Verteidigungstürmen ziehen Lebenspunkte von diesem Wert ab; fällt er
         unter Null, so stirbt diese Einheit (\refid{A:die}). \\
    AS & Angriffsstärke
       & Schaden, den diese Einheit am gegenerischen Lager verursacht, wenn sie
         dieses erreicht (\refid{A:damage-base}). \\
    GS & Geschwindigkeit & Distanz, die pro Zeiteinheit zurückgelegt werden
         kann. \\

    \bottomrule
  \end{longtabu}
  \todo[inline, caption = {Größe und Kollision}]{%
    Größe und Kollision evtl. in die Tabelle aufnehmen, aber was sagt die Größe
    genau aus? Diese ist doch nur interessant, wenn die Einheit kollidiert?}
\end{table}

\begingroup
  \small
  \begin{longtabu}{rXp{0.191\linewidth}}
    \rowfont{\normalsize}
    \caption{Truppen und ihre Werte\label{tab:attack-units}}
    \\\midrule[\heavyrulewidth]\endfirsthead

    % TODO: This seems to introduce too much vertical whitespace between the
    % midrule and the next row.
    \rowfont{\normalsize}
    \caption[]{Truppen und ihre Werte (fortges.)}
    \\\midrule[\heavyrulewidth]\endhead

    % TODO: At the moment this leads to strange page breaks, revisit when there
    % is more content!
    %
    % \multicolumn{3}{r}{\itshape fortges. auf der nächsten Seite}
    % \\\endfoot
    %
    % \endlastfoot

    \multicolumn{3}{c}{\bfseries Bug} \\*\midrule
    B  & Eine schnelle Sprintereinheit ohne viele Lebenspunkte, die alleine
         nicht besonders viel Schaden verursacht, aber in großer Masse gekauft
         werden kann, da sie nicht viel kostet.
       & \itshape Bild hier? \\*
    K  & 1    \\*
    LP & 1    \\*
    AS & 1    \\*
    GS & 10   \\
    \midrule[\heavyrulewidth]

    \multicolumn{3}{c}{\bfseries Virus} \\*\midrule
    B  & Durchschnittliche Einheit, die etwas mehr kostet als ein \emph{Bug,}
         etwas langsamer ist, aber mehr LP hat und mehr Schaden verursacht.
       & \itshape Bild hier? \\*
    K  & 2      \\*
    LP & 2      \\*
    AS & 2      \\*
    GS & 5      \\
    \midrule[\heavyrulewidth]

    \multicolumn{3}{c}{\bfseries Trojaner} \\*\nopagebreak\midrule\nopagebreak
    B  & Stirbt diese Einheit, werden an der Stelle ihres Todes \emph{Bugs} und
         \emph{Viren} gespawnt. Ein Trojaner ist zwar relativ langsam und kostet
         mehr als \emph{Viren,} hat dafür aber mehr LP und mehr AS.
       & \itshape Bild hier? \\*
    K  & 4 \\*
    LP & 4 \\*
    AS & 4 \\*
    GS & 3 \\
    \midrule[\heavyrulewidth]

    \multicolumn{3}{c}{\bfseries Nokia} \\*\midrule
    B  & Diese Einheit ist bei gleichen Kosten zwar langsamer als ein
         \emph{Trojaner,} dafür aber hat sie mehr LP und AS.
       & \itshape Bild hier? \\*
    K  & 4 \\*
    LP & 6 \\*
    AS & 6 \\*
    GS & 2 \\
    \midrule[\heavyrulewidth]

    \multicolumn{3}{c}{\bfseries Thunderbird} \\*\midrule
    B  & Diese Einheit fliegt, daher muss sie nicht den Weg um Mauern und Türme
         herumfinden, sondern kann einfach auf Luftlinie darüber hinwegfliegen.

         Von den Kosten ist diese Einheit mit \emph{Trojaner} vergleichbar, sie
         ist zwar etwas schneller, hat aber nicht viele LP und weniger AS.
       & \itshape Bild hier? \\*
    K  & 4 \\*
    LP & 4 \\*
    AS & 3 \\*
    GS & 4 \\

    \bottomrule
  \end{longtabu}
  \todo[inline]{Bilder für Truppen}
\endgroup

\begingroup
  \small
  \begin{longtabu}{rXp{0.191\linewidth}}
    \rowfont{\normalsize}
    \caption{Helden und ihre Werte\label{tab:attack-heroes}}
    \\\midrule[\heavyrulewidth]\endfirsthead

    % TODO: This seems to introduce too much vertical whitespace between the
    % midrule and the next row.
    \rowfont{\normalsize}
    \caption[]{Helden und ihre Werte (fortges.)}
    \\\midrule[\heavyrulewidth]\endhead

    % TODO: At the moment this leads to strange page breaks, revisit when there
    % is more content!
    %
    % \multicolumn{3}{r}{\itshape fortges. auf der nächsten Seite}
    % \\\endfoot
    %
    % \endlastfoot

    \multicolumn{3}{c}{\bfseries Settings} \\*\midrule
    B  & Diese Einheit heilt Truppen um sich herum, hat jedoch selbst
         eher wenig LP; diese ist die langsamste der Heldeneinheiten, sie
         verursacht am gegenerischen Lager keinen Schaden.
       & \itshape Bild hier? \\*
    F  & \emph{(passiv)} heilt verbündete Truppen in einem gewissen Radius
         regelmäßig um einen Wert (\refid{H:heal}).\\*
    K  & 10   \\*
    LP & 4    \\*
    AS & 0    \\*
    GS & 4    \\
    \midrule[\heavyrulewidth]

    \multicolumn{3}{c}{\bfseries Firefox} \\*\midrule
    B  & Dieser Held ist eine starke Angriffseinheiten, die mit ihrer Fähigkeit
         leichter zwischen den Verteidigungsgebäuden hindurchkommt. Der
         \emph{Firefox} ist relativ schnell, hat durchschnittliche LP und
         relativ viel~AS.
       & \itshape Bild hier? \\*
    LP & 6      \\*
    AS & 8      \\*
    GS & 8      \\*
    K  & 10     \\*
    F  & \emph{(aktiv)} kann Verteidigungsgebäude überspringen
         (\refid{H:jump}).\\
    \midrule[\heavyrulewidth]

    \multicolumn{3}{c}{\bfseries Bluescreen} \\*\nopagebreak\midrule\nopagebreak
    B  & Diese Einheit unterstützt verbündete Einheit, indem sie gegenerische
         Verteidigungsgebäude für einen Moment deaktivieren kann; dafür
         verursacht sie am gegenerischen Lager selbst keinen Schaden, hat wenige
         LP ist aber schnell.
       & \itshape Bild hier? \\*
    LP & 4      \\*
    AS & 0      \\*
    GS & 10     \\*
    K  & 10     \\*
    F  & \emph{(aktiv)} kann eine Schockwelle zünden, um gegenerische
         Verteidigungsgebäude in der Nähe für einen Moment zu deaktivieren
         (\refid{H:emp}).

         Um diese Fähigkeit erneut einzusetzen, muss diese Einheit zur Basis
         zurückkehren um sich aufzuladen (\refid{H:reload}).\\

    \bottomrule
  \end{longtabu}
  \todo[inline]{Bilder für Helden}
\endgroup
