\section{Spielobjekte}

% Hier sind alle Spielobjekte aufgelistet, die es im Spiel gibt. Dies können
% zum Beispiel Einheiten, Gebäude, Hindernisse usw. sein. Hilfreich zum
% Verständnis und zur Identifizierung der einzelnen Objekte können Bilder sein.
%
% Insbesondere erklärt dieser Abschnitt alle Eigenschaften und Fähigkeiten der
% unterschiedlichen Objekte. Sind sie zum Beispiel einfach zu zerstören, kosten
% sie viele Ressourcen, usw.
%
% Spielobjekte können hier auch schon mit konkreten Werten versehen werden;
% wenn noch keine konkreten Werte vorliegen (z.B. wie viele Lebenspunkte eine
% bestimmte Einheit hat), sollten zumindest die Verhältnisse der Werte von
% unterschiedlichen Spileobjekten aufgelistet werden, also zum Beispiel: Welche
% Einheit ist stärker als eine andere? Dass sich konkrete Werte noch verändern
% können, ist selbstverständlich, und eine Frage des Balancings gegen Ende des
% Projekts.

\missingSection{Spielobjekte}

\todo[inline, caption = {Seitenumbrüche}]{%
  Wenn der vorhergehende Teil des Kapitels fertig ist, müssen evtl.
  Seitenumbrüche manuell eingefügt werden, um ein sinnvolles Ergebnis zu
  erzielen.}

\subsection{Angriffseinheiten}

\begin{description}
  \item[Truppen]\todo[noline]{Benennung überdenken}
    kosten relativ wenig, lassen sich jedoch nicht weiter kontrollieren. Diese
    Einheiten verfolgen das Ziel, möglichst schnell zum gegenerischen Lager zu
    gelangen um dort Schaden zu verursachen.

  \item[Helden] kosten mehr als Truppen, diese Einheiten lassen sich jedoch vom
    Spieler kontrollieren und so strategisch einsetzen und außerhalb der
    Reichweite von Verteidigungsgebäuden positionieren; zusätzlich besitzen sie
    Fähigkeiten, die der Spieler einsetzen kann.

\end{description}

Tabelle~\ref{tab:attack-unit-props} beschreibt die Eigenschaften die
Angriffseinheiten haben, in Tabelle~\ref{tab:attack-units} sind alle Truppen
mit ihren Eigenschaften aufgelistet und Tabelle~\ref{tab:attack-heroes} enthält
alle Helden.

\begin{table}[htbp]
  \caption{Eigenschaften von Angriffseinheiten}
  \label{tab:attack-unit-props}
  \small
  \begin{longtabu}{rlX}
    \toprule\rowfont{\itshape}
    & Eigenschaft & Beschreibung \\
    \midrule

    B  & Beschreibung
       & Eine allgemeine Beschreibung dieser Einheit und Vergleich zu anderen
         Einheiten. \\
    F  & Fähigkeit
       & Nur Helden haben eine Fähigkeit, diese kann vom Spieler aktiviert
         werden (\refid{A:hero-ability}). \\
    K  & Kosten
       & Die Menge an Gold die aufgewendet werden muss, um eine dieser
         Einheiten zu kaufen (\refid{A:buy-attack}). \\
    LP & Lebenspunkte
       & Die Zahl der Lebenspunkte einer Einheit: Angriffe von
         Verteidigungstürmen ziehen Lebenspunkte von diesem Wert ab; fällt er
         unter Null, so stirbt diese Einheit (\refid{A:die}). \\
    AS & Angriffsstärke
       & Schaden, den diese Einheit am gegenerischen Lager verursacht, wenn sie
         dieses erreicht (\refid{A:damage-base}). \\
    GS & Geschwindigkeit & Distanz, die pro Zeiteinheit zurückgelegt werden
         kann. \\

    \bottomrule
  \end{longtabu}
  \todo[inline, caption = {Größe und Kollision}]{%
    Größe und Kollision evtl. in die Tabelle aufnehmen, aber was sagt die Größe
    genau aus? Diese ist doch nur interessant, wenn die Einheit kollidiert?}
\end{table}

\begingroup
  \small
  \begin{longtabu}{rXp{0.191\linewidth}}
    \rowfont{\normalsize}
    \caption{Truppen und ihre Werte\label{tab:attack-units}}
    \\\midrule[\heavyrulewidth]\endfirsthead

    % TODO: This seems to introduce too much vertical whitespace between the
    % midrule and the next row.
    \rowfont{\normalsize}
    \caption[]{Truppen und ihre Werte (fortges.)}
    \\\midrule[\heavyrulewidth]\endhead

    % TODO: At the moment this leads to strange page breaks, revisit when there
    % is more content!
    %
    % \multicolumn{3}{r}{\itshape fortges. auf der nächsten Seite}
    % \\\endfoot
    %
    % \endlastfoot

    \multicolumn{3}{c}{\bfseries Bug} \\*\midrule
    B  & Eine schnelle Sprintereinheit ohne viele Lebenspunkte, die alleine
         nicht besonders viel Schaden verursacht, aber in großer Masse gekauft
         werden kann, da sie nicht viel kostet.
       & \missingpic \\*
    K  & 1    \\*
    LP & 1    \\*
    AS & 1    \\*
    GS & 10   \\
    \midrule[\heavyrulewidth]

    \multicolumn{3}{c}{\bfseries Virus} \\*\midrule
    B  & Durchschnittliche Einheit, die etwas mehr kostet als ein \emph{Bug,}
         etwas langsamer ist, aber mehr LP hat und mehr Schaden verursacht.
       & \missingpic \\*
    K  & 2      \\*
    LP & 2      \\*
    AS & 2      \\*
    GS & 5      \\
    \midrule[\heavyrulewidth]

    \multicolumn{3}{c}{\bfseries Trojaner} \\*\nopagebreak\midrule\nopagebreak
    B  & Stirbt diese Einheit, werden an der Stelle ihres Todes \emph{Bugs} und
         \emph{Viren} gespawnt. Ein Trojaner ist zwar relativ langsam und kostet
         mehr als \emph{Viren,} hat dafür aber mehr LP und mehr AS.
       & \missingpic \\*
    K  & 4 \\*
    LP & 4 \\*
    AS & 4 \\*
    GS & 3 \\
    \midrule[\heavyrulewidth]

    \multicolumn{3}{c}{\bfseries Nokia} \\*\midrule
    B  & Diese Einheit ist bei gleichen Kosten zwar langsamer als ein
         \emph{Trojaner,} dafür aber hat sie mehr LP und AS.
       & \missingpic \\*
    K  & 4 \\*
    LP & 6 \\*
    AS & 6 \\*
    GS & 2 \\
    \midrule[\heavyrulewidth]

    \multicolumn{3}{c}{\bfseries Thunderbird} \\*\midrule
    B  & Diese Einheit fliegt, daher muss sie nicht den Weg um Mauern und Türme
         herumfinden, sondern kann einfach auf Luftlinie darüber hinwegfliegen.

         Von den Kosten ist diese Einheit mit \emph{Trojaner} vergleichbar, sie
         ist zwar etwas schneller, hat aber nicht viele LP und weniger AS.
       & \missingpic \\*
    K  & 4 \\*
    LP & 4 \\*
    AS & 3 \\*
    GS & 4 \\

    \bottomrule
  \end{longtabu}
  \missingpics{Bilder für Truppen}
\endgroup

\begingroup
  \small
  \begin{longtabu}{rXp{0.191\linewidth}}
    \rowfont{\normalsize}
    \caption{Helden und ihre Werte\label{tab:attack-heroes}}
    \\\midrule[\heavyrulewidth]\endfirsthead

    % TODO: This seems to introduce too much vertical whitespace between the
    % midrule and the next row.
    \rowfont{\normalsize}
    \caption[]{Helden und ihre Werte (fortges.)}
    \\\midrule[\heavyrulewidth]\endhead

    % TODO: At the moment this leads to strange page breaks, revisit when there
    % is more content!
    %
    % \multicolumn{3}{r}{\itshape fortges. auf der nächsten Seite}
    % \\\endfoot
    %
    % \endlastfoot

    \multicolumn{3}{c}{\bfseries Settings} \\*\midrule
    B  & Diese Einheit heilt Truppen um sich herum, hat jedoch selbst
         eher wenig LP; diese ist die langsamste der Heldeneinheiten, sie
         verursacht am gegenerischen Lager keinen Schaden.
       & \missingpic \\*
    F  & \emph{(passiv)} heilt verbündete Truppen in einem gewissen Radius
         regelmäßig um einen Wert (\refid{H:heal}).\\*
    K  & 10   \\*
    LP & 4    \\*
    AS & 0    \\*
    GS & 4    \\
    \midrule[\heavyrulewidth]

    \multicolumn{3}{c}{\bfseries Firefox} \\*\midrule
    B  & Dieser Held ist eine starke Angriffseinheiten, die mit ihrer Fähigkeit
         leichter zwischen den Verteidigungsgebäuden hindurchkommt. Der
         \emph{Firefox} ist relativ schnell, hat durchschnittliche LP und
         relativ viel~AS.
       & \missingpic \\*
    LP & 6      \\*
    AS & 8      \\*
    GS & 8      \\*
    K  & 10     \\*
    F  & \emph{(aktiv)} kann Verteidigungsgebäude überspringen
         (\refid{H:jump}).\\
    \midrule[\heavyrulewidth]

    \multicolumn{3}{c}{\bfseries Bluescreen} \\*\nopagebreak\midrule\nopagebreak
    B  & Diese Einheit unterstützt verbündete Einheit, indem sie gegenerische
         Verteidigungsgebäude für einen Moment deaktivieren kann; dafür
         verursacht sie am gegenerischen Lager selbst keinen Schaden, hat wenige
         LP ist aber schnell.
       & \missingpic \\*
    LP & 4      \\*
    AS & 0      \\*
    GS & 10     \\*
    K  & 10     \\*
    F  & \emph{(aktiv)} kann eine Schockwelle zünden, um gegenerische
         Verteidigungsgebäude in der Nähe für einen Moment zu deaktivieren
         (\refid{H:emp}).

         Um diese Fähigkeit erneut einzusetzen, muss diese Einheit zur Basis
         zurückkehren um sich aufzuladen (\refid{H:reload}).\\

    \bottomrule
  \end{longtabu}
  \missingpics{Bilder für Helden}
\endgroup


\subsection{Verteidigungsgebäude}

In Tabelle~\ref{tab:defend-props} werden die Eigenschaften von
Verteidigungsgebäuden beschrieben, Tabelle~\ref{tab:defend-units} enthält die
Gebäude und weist den Eigenschaften Werte zu. Der Wert W berechnet sich aus der
Menge an Bitcoin, die in dieses Gebäude investiert wurde.

Im Laufe des Spiels kann der Spieler folgende Aktionen auf eigenen Türmen
ausführen:

\begin{description}
  \item[Verkaufen] (\refid{A:tower-sell}) Das Gebäude verschwindet, es können
    neue Gebäude an dieser Stelle gebaut werden und feindliche Einheiten können
    wieder über diese Felder laufen.

    Der Spieler erhält 80\,\% des Gebäudewertes an Bitcoin.

  \item[Verbessern] (\refid{A:tower-improve}) Erhöht die Reichweite des
    Gebäudes (außer bei \emph{Schockfeld}) um 50\,\% des aktuellen Wertes und
    reduziert das Angriffsintervall um 20\,\% des aktuellen Wertes.

    Die Verbesserung kostet den Spieler 50\,\% des aktuellen Turmwertes und der
    Turmwert steigt um diese Kosten.

    Bei \emph{Kabel} ist keine Verbesserung möglich, jedes andere Gebäude kann
    maximal zweimal verbessert werden.

  \item[Strategie wählen] (\refid{A:tower-strategy}) Mögliche Strategien sind
    \begin{description}[itemsep=0pt]
      \item[Erste Einheit] \emph{(ist standardmäßig ausgewählt)}\\
        Greift die Einheit an, die den kürzesten Weg hat, um Schaden an der Basis
        zu verusachen.

      \item[Stärkste Einheit] ~\\
        Greift die Einheit an, die die meisten LP hat.

      \item[Schwächste Einheit] ~\\
        Greift die Einheit an, die die wenigsten LP hat.
    \end{description}

    Bei \emph{Kabel} und \emph{Schockfeld} ist ein wählen der Strategie nicht
    möglich.

\end{description}


\begingroup
  \small
  \begin{longtabu}{rlX}
    \rowfont{\normalsize}
    \caption{Eigenschaften von Verteidigungsgebäuden\label{tab:defend-props}}\\

    \midrule[\heavyrulewidth]\rowfont{\itshape}
    & Eigenschaft & Beschreibung \\
    \midrule

    B  & Beschreibung
       & Eine allgemeine Beschreibung dieser Einheit und Vergleich zu anderen
         Einheiten. \\
    K  & Kosten
       & Die Menge an Bitcoin die aufgewendet werden muss, um eines dieser
         Gebäude zu platzieren~(\refid{A:put-defend}). \\
    VS & Verteidigungsstärke
       & Schaden, den dieses Gebäude an getroffenen Gegner
         verursacht~(\refid{A:unit-hit}). \\
    AI & Angriffsintervall
       & Zeit die vergehen muss, bevor dieses Gebäude erneut Gegner angreifen
         kann~(\refid{A:tower-attack}). \\
    RW & Reichweite
       & Radius um den Turm, in dem Einheiten angegriffen werden können, und in
         dem die Effekte der Türme auf die Einheiten wirken. \\

    \bottomrule
  \end{longtabu}
\endgroup

\begingroup
  \small
  \begin{longtabu}{rXp{0.191\linewidth}}
    \rowfont{\normalsize}
    \caption{Verteidigungsgebäude und ihre Werte\label{tab:defend-units}}
    \\\midrule[\heavyrulewidth]\endfirsthead

    \rowfont{\normalsize}
    \caption[]{Verteidigungsgebäude und ihre Werte (fortges.)}
    \\\midrule[\heavyrulewidth]\endhead

    \multicolumn{3}{r}{\itshape fortges. auf der nächsten Seite}
    \\\endfoot

    \endlastfoot

    \multicolumn{3}{c}{\bfseries Kabel} \\*\midrule
    B  & Dieses Gebäude kostet wenig, steht gegnerischen Einheiten im Weg und
         verursacht keinen Schaden.
       & \missingpic \\*
    K  & 2 \\*
    VS & --- \\*
    AI & --- \\*
    RW & --- \\
    \midrule[\heavyrulewidth]

    \multicolumn{3}{c}{\bfseries Mauszeigerschütze} \\*\midrule
    B  & Durchschnittlicher Verteidigungsturm, der Mauszeiger auf ein
         Einzelziel verschießt.
       & \missingpic \\*
    K  & 3 \\*
    VS & 1 \\*
    AI & 1 \\*
    RW & 4 \\
    \midrule[\heavyrulewidth]

    \multicolumn{3}{c}{\bfseries CD-Werfer} \\*\midrule
    B  & Dieser Turm kostet mehr und schießt langsamer als ein
         \emph{Mauszeigerschütze,} dafür verursacht das Projektil (die CD)
         jedoch auf ihrem Weg an jedem berührten Gegner den Schaden der Höhe
         VS.
       & \missingpic \\*
    K  & 5 \\*
    VS & 4 \\*
    AI & 3 \\*
    RW & 3 \\
    \midrule[\heavyrulewidth]

    \multicolumn{3}{c}{\bfseries Antivirusprogramm} \\*\midrule
    B  & Von den Kosten ist dieser Turm vergleichbar zum \emph{CD-Werfer,}
         allerdings schießt das \emph{Antivirusprogramm} noch langsamer,
         verursacht dafür aber an einem Einzelziel erheblichen Schaden.
       & \missingpic \\*
    K  & 5 \\*
    VS & 7 \\*
    AI & 5 \\*
    RW & 6 \\
    \midrule[\heavyrulewidth]

    \multicolumn{3}{c}{\bfseries Lüftung} \\*\midrule
    B  & Dieser Turm verlangsamt alle Einheiten im Einflussbereich.
       & \missingpic \\*
    K  & 5 \\*
    VS & 7 \\*
    AI & 5 \\*
    RW & 4 \\
    \midrule[\heavyrulewidth]

    \multicolumn{3}{c}{\bfseries Wifi-Router} \\*\midrule
    B  & Dieser Turm schießt nahezu dauerhaft kreisförmige Wellen, die wenig
         Schaden verursachen und Gegner penetrieren.
       & \missingpic \\*
    K  & 5 \\*
    VS & 2 \\*
    AI & 1 \\*
    RW & 5 \\
    \midrule[\heavyrulewidth]

    \multicolumn{3}{c}{\bfseries Schockfeld} \\*\midrule
    B  & Dieses „Gebäude“ blockiert die Gegner nicht, sie laufen darüber
         hinweg. In regelmäßigen Abständen erhalten alle Gegner schaden, die
         auf einem \emph{Schockfeld} sind.
       & \missingpic \\*
    K  & 4 \\*
    VS & 2 \\*
    AI & 3 \\*
    RW & 0 \\

    \bottomrule
  \end{longtabu}
  \missingpics{Bilder für Verteidigungsgebäuden}
\endgroup


\subsection{Upgrades}\label{sec:upgrades-list}

% Folgende Zeilen gehören eher in die Beschreibung des Interfaces und der Spielstruktur
%
% Die Upgrades-Tabelle befindet sich in der Mitte des Spielfeldes zwischen den 2 Strecken.
% \begin{itemize}
% \item Die Tabelle ist in mehreren Spalten aufgeteilt, Upgrades aus der nächster Spalte kosten mehr Erfahrungspunkte als die Upgrades von den
% Spalten die davor kamen. (erste Spalte ist die allerlinkste, letzte Spalte ist die allerrechteste)
% \item Upgrades kann man jederzeit mit Erfahrungspunkten kaufen, die bekommt man am Ende jeder Welle die man überlebt hat.
% \item Die 2 Spieler teilen diese Upgrades-Tabelle, wenn einer von ihnen ein Upgrade gekauft hat, kann der
% andere es nicht mehr bekommen, seine Kaufentscheidungen muss man also gut planen.
% \end{itemize}

Mit Erfahrungspunkten (EP), die der Spieler durch das Besiegen gegnerischer
Wellen erhält, kann er sich Upgrades kaufen.

\begin{description}
  \item[Kosten: 1 EP]~
    \begin{itemize}[nosep, leftmargin=0.2cm]
      \item LP aller Einheiten um 5\,\% verbessern.
      \item GS aller Einheiten um 5\,\% erhöhen.
      \item VS aller Gebäude um 5\,\% verbessern.
      \item AI aller Gebäude um 5\,\% reduzieren.
    \end{itemize}

  \item[Kosten: 2 EP]~
    \begin{itemize}[nosep, leftmargin=0.2cm]
      \item LP aller Einheiten 10\,\% verbessern.
      \item VS aller Gebäude 10\,\% verbessern.
      \item GS aller Einheiten um 10\,\% erhöhen.
      \item 10\,\% mehr Bitcoin pro Sekunde.
    \end{itemize}

  \item[Kosten: 3 EP]~
    \begin{itemize}[nosep, leftmargin=0.2cm]
      \item \emph{CD-Turm} schießt CD's als Boomerang.
	    \item GS von \emph{Nokia} um 40\,\% erhöhen.
	    \item GS von \emph{Firefox} wird um 10\,\% erhöht.
	    \item \emph{Trojaner} transportieren 5 Einheiten mehr.

    \end{itemize}

  \item[Kosten: 4 EP]~
    \begin{itemize}[nosep, leftmargin=0.2cm]
      \item Möglichkeit, bis zu zwei \emph{Firefox}-Einheiten gleichzeitig zu
        kontrollieren.
	    \item EMP-Effekt von \emph{Bluescreen} dauert 50\,\% länger.
	    \item \emph{Trojaner} transportieren 10 Einheiten mehr.
	    \item Einzugsbereich von \emph{Settings} um 5\,\% größer.
	    \item Heil-Rate von \emph{Settings} um 5\,\% erhöhen.

    \end{itemize}

  \pagebreak
  \item[Kosten: 5 EP]~
    \begin{itemize}[nosep, leftmargin=0.2cm]
      \item Möglichkeit, bis zu zwei \emph{Firefox}-Einheiten gleichzeitig zu
        kontrollieren.
      \item \emph{Bluescreen} hat einen zweiten EMP-Angriff, bevor eine
        Aufladung nötig ist.
	    \item Einzugsbereich von \emph{Settings} um 10\,\% größer.
	    \item Heil-Rate von \emph{Settings} um 10\,\% erhöhen.
    \end{itemize}

\end{description}


\subsection{Basis}

Die Basis eines Spielers erfüllt folgende Funktionen.

\begin{itemize}
  \item Zu Beginn des Spiels eine Ladung (L) von 100\,\%, fällt die Ladung auf
    0\,\% oder weniger, so geht die Basis kaputt (\refid{A:base-die}).

  \item Die Basis ist das Ziel der feindlichen Angriffseinheiten; wird sie von
    diesen erreicht, so werden von der Ladung so viel Prozent abgezogen wie die
    Einheit AS hat (\refid{A:damage-base}).

  \item Kehrt der \emph{Bluescreen} zur Basis zurück, kann er erneut seine
    Fähigkeit einsetzen (\refid{H:reload}).

  \item Neue Angriffseinheiten, die der Spieler kauft (\refid{A:buy-attack})
    spawnen bei der Basis und begeben sich auf den Weg zur gegenerischen Basis.

\end{itemize}
