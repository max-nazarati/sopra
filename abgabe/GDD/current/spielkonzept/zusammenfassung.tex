\section{Zusammenfassung}

% Hier ist ein kurzer einleitender Text evtl. in Verbindung mit einem Bild
% gefragt. Ziel ist es, das zu erstellende Spiel in kurzen Sätzen zu erklären
% und die Grundidee zu erläutern. Für die Zusammenfassung kann man sich an den
% "Klappentexten" auf der Rückseite von Spieleverpackungen orientieren. Der
% Text darf als einziger im GDD auch reißerisch und dramatisch sein (abgesehen
% vom Screenplay).

\textit{Kernel Panic!} ist ein a-mazing tower defense Spiel.
Du bist ein Supercomputer und wirst von einem fießen Hacker angegriffen, der versucht deinen Akku in die Knie zu zwingen.
Also handle schnell und baue dir eine geschickte 'Firewall' auf bevor du zwangsweise in den Ruhezustand versetzt wirst!
Dazu kannst du unter anderem dein Hardware-Lager plündern und mit defekten Geräten den Weg versperren.
Doch dein Gegner wird keine Ruhe geben, bevor du nicht vernichtet bist.
Zum Glück hast du Connections zu einem russichen Hacker-Collectiv, das dir Trojaner, Viren und Co verkaufen kann. Vergeude keine Zeit und schicke in Upload-Wellen deine eigenen Bug-Armeen los um den Gegner mit seinen eigenen Waffen zu schlagen.
Auch Firefox, Thunderbird und viele mehr können dir zu Hilfe eilen und gezielt angreifen.
Mit der Zeit sammelst du wichtige Erfahrungen und kannst so deine Angriffe upgraden. Aber mach schnell, denn auch dein Gegner ist auf Upgrades aus und könnte sie dir wegschnappen.
