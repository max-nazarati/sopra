\documentclass[version = last]{scrreprt}

%%%%%%%%%%%%%%%%%%%%%%%%%%%%%%%%%%%%%%%%%%%%%%%%%%%%%%%%%%%%%%%%%%%%%%%%%%%%%%%
% To create the final version, uncomment the line containing \finaltrue
\newif\iffinal
\finalfalse
% \finaltrue


%%%%%%%%%%%%%%%%%%%%%%%%%%%%%%%%%%%%%%%%%%%%%%%%%%%%%%%%%%%%%%%%%%%%%%%%%%%%%%%
% Packages & Options

\usepackage[utf8]{inputenc}
\usepackage[T1]{fontenc}
\usepackage[ngerman]{babel}
\usepackage{microtype}
\usepackage{lmodern}
\usepackage{graphicx}
\usepackage{hyperref}
\usepackage{booktabs, longtable, tabu}
\usepackage{enumitem}

\iffinal
  % Hide todos in the final version.
  \usepackage[final, obeyFinal]{todonotes}
\else
  \usepackage[ngerman, textsize=footnotesize]{todonotes}
\fi

\iffinal
  % Don't show overfull paragraphs in the final version.
\else
  % This option helps finding paragraphs which are wider than linewidth by
  % drawing a black bar on the right side.
  \KOMAoption{overfullrule}{true}
\fi


%%%%%%%%%%%%%%%%%%%%%%%%%%%%%%%%%%%%%%%%%%%%%%%%%%%%%%%%%%%%%%%%%%%%%%%%%%%%%%%
% Custom commands & adjustments

% Commands to indicate missing parts.
\newcommand\missingChapter[2][]{\todo[inline, #1]{\thechapter{} „#2“ schreiben}}
\newcommand\missingSection[2][]{\todo[inline, #1]{\thesection{} „#2“ schreiben}}

% Decrease the font size slightly from \Large to fit all names in two lines.
\addtokomafont{author}{\large}

% Decrease space between table caption and the table's top-rule.
\setlength{\abovetopsep}{-0.8em}


\begin{document}

\title{Spielname\todo[noline]{Das Spiel braucht einen Namen}}
\author{%
  % Sorted alphabetically by last name.
  Melissa Hägle,
  Zacharias Häringer,
  Johannes Mannhardt,\\
  Maximilian Nazarati,
  Jens Rahnfeld,
  Zo\"e Schaaff,
  Janek Spaderna}
\subject{Gruppe 09}
\publishers{Tutor: Daniel Lux}
\maketitle


\tableofcontents
\addcontentsline{toc}{chapter}{Inhaltsverzeichnis}


\iffinal
  % Hide the list of TODOs in the final version.
\else
  \listoftodos
  \addcontentsline{toc}{chapter}{TODOs}
\fi


%%%%%%%%%%%%%%%%%%%%%%%%%%%%%%%%%%%%%%%%%%%%%%%%%%%%%%%%%%%%%%%%%%%%%%%%%%%%%%%
% Main Content starts here

\chapter{Spielkonzept}

\section{Zusammenfassung}

% Hier ist ein kurzer einleitender Text evtl. in Verbindung mit einem Bild
% gefragt. Ziel ist es, das zu erstellende Spiel in kurzen Sätzen zu erklären
% und die Grundidee zu erläutern. Für die Zusammenfassung kann man sich an den
% "Klappentexten" auf der Rückseite von Spieleverpackungen orientieren. Der
% Text darf als einziger im GDD auch reißerisch und dramatisch sein (abgesehen
% vom Screenplay).

\missingSection{Zusammenfassung}


\section{Alleinstellungsmerkmal}

% Was hebt dieses Spiel von der Masse ab? Wodurch versucht man, den Spieler
% (und den Kunden) zu begeistern? Das Alleinstellungsmerkmal ist das Merkmal
% des Spiels, welches es einzigartig macht. Ein Alleinstellungsmerkmal kann
% sowohl ein Feature, als auch ein gesamtes Konzept des Spiels sein.

\missingSection{Alleinstellungsmerkmal}



\chapter{Benutzeroberfläche}

% Hier wird erklärt, mit welchen Eingaben und über welche Steuerelemente der
% Spieler mit dem Spiel interagiert.


Achtung! In diesem Abschnitt werden viele Spielobjekte, die bisher noch nicht genau beschrieben wurden, gezeigt. Es kann für einen ersten Eindruck sinnvoll sein, sich die Benutzeroberfläche vorher anzuschauen. Um das Kapitel vollständig zu verstehen, wird empfohlen, an diese Stelle zurückzukehren, sobald man sich mit der Spiellogik auseinandergesetzt hat.
\section{Spieler-Interface}
% Aus dem Wiki
% ------------
%
% Siehe: https://sopra.informatik.uni-freiburg.de/soprawiki/index.php?title=GDD#Spieler-Interface
%
% Dieser Abschnitt beinhaltet eine Beschreibung des Spielbildschirms, also
% dessen, was für den Spieler sichtbar ist. Dies beinhaltet die Art der
% Darstellung (2D oder 3D), die Kamerasicht, usw.
%
% Wichtig ist, dass alle sichtbaren Elemente, wie Minimap, Menüleiste, etc.
% erklärt werden. Durch ein Bild eines typischen Vertreters dieser Spielart
% oder durch eine Konzeptzeichnung des Interfaces kann die Beschreibung noch
% verbessert werden. In der finalen Version des GDDs können auch Screenshots
% des eigenen Spiels verwendet werden.
%
% Außerdem wird in diesem Abschnitt erklärt, wie der Spieler das Spiel steuert
% (mit der Maus, mit Maus und Tastatur, Joystick, Gamepad usw.). Alle Aktionen,
% die der Spieler durchführen kann, müssen erklärt werden. Auch mögliche
% Shortcuts und/oder Tastenkombinationen sollten hier erwähnt werden.
%
%
% Aus den Folien
% --------------
%
% Siehe: https://sopra.informatik.uni-freiburg.de/soprawiki/images/1/11/How-To-GDD_SS19.pdf
%
% * Beschreibung dessen, was der Spieler sieht
% * Art der Darstellung, Kameraansichten, sichtbare Elemente (HUD, Minimap,
%   Menüleisten, usw.)
% * Bild (Konzeptzeichnung, Screenshot, Mockup) dessen, wie das Spiel aussehen
%   soll.
% * Beschreibung der Steuerung.
\begin{figure}[ht]
	\centering
	\includegraphics[width=1\textwidth]{spieler-interface.png}
	\caption{Spieler-Interface}
	\label{fig:spieler-interface}
\end{figure}
\pagebreak\noindent
Abbildung \ref{fig:spieler-interface} zeigt eine typische Momentaufnahme des Spiels \textit{Kernel Panic!}.
Der Spieler betrachtet die Spielwelt aus der Top-Down Perpektive.

Beschreiben wir zunächst das Head-up-Display (HUD) bestehend aus den rot markierten Bereichen mit den Nummerierungen $1$ bis $7$:
\begin{itemize}[leftmargin=*]
	\item{1 Spielstand:} Hier werden die wichtigsten Informationen angezeigt um im Spiel den Überblick zu behalten, vergleichbar mit einem Punktestand.
	\begin{itemize}
		\item{$1 A$: Die aktuelle \textit{Spielzeit} zeigt an, wieviele Minuten und Sekunden das aktuelle Spiel bereits im Gange ist.}
		\item{$1 B$: Direkt unter der \textit{Spielzeit} ist die Anzahl der erfolgreich besiegten \textit{Wellen} im Überblick zu sehen: links die eigenen, rechts die des Gegners.}
		\item{$1 C_{1}$ \& $1C_{2}$:} $C_{1}$ zeigt die \textit{Erfahrungspunkte,} die man sich erspielt hat, $C_{2}$ die des Gegners.
		\item{$D_{1}$ \& $D_{2}$:} Die aktuelle \emph{Ladung} der Spieler. Hier gilt ebenfalls, links die eigenen auf der rechten Seite die des Gegners.
	\end{itemize}
	\item {$2$ Verteidigungsgebäude:} Oben, am linken Bildrand ist eine Liste an \textit{Verteidigungsgebäuden}. Um die verschiedenen Gebäude auszuwählen kann man die Tafeln $2 A$ bis $2 E$ benutzen.
	\item {$3$ \& $4$ Angriffseinheiten:} Auf der gegenüberliegenden Seite des Bildschirms befinden sich die \textit{Angriffseinheiten}. Dabei ist $3$ die Liste der \textit{Truppen} und $4$ die Liste der \textit{Helden}. Auch hier kann man mit den verschiedenen Tafeln ($3 A$ bis $3 E$ bzw $4 A$ bis $4 C$) die genaue Auswahl treffen.
	\item {$5$ Übersicht:}
	\begin{itemize}
		\item{$5 B$ Pause: Mithilfe von diesem Feld kann man das Spiel pausieren und auf die Menüstruktur (siehe Abschnitt~\ref{sec:menu}) zugreifen.}
		\item{$5 A$: Im unteren rechten Eck ist eine Mini-map, wie man sie aus MOBAs oder Strategiespielen kennt. Sie zeigt abstrahiert die gesamte Spielwelt.}
		\item{$5 C$: der aktuelle Kameraausschnitt ist auf der Mini-map markiert; man kann den Bildausschnitt symbolisch erkennen.}
	\end{itemize}
	\item {$6$ Auswahl-Tafel: Zeigt Informationen über ein aktuell ausgewähltes Spielobjekt.}	\begin{itemize}
		\item{$6 A$ bis $6 C$: Der hier ausgewählte Mauszeigerschütze hat drei mögliche Aktionen: \textbf{Verbessern ($6 A$)}, \textbf{Verkaufen ($6 B$)} und \textbf{Strategie wählen ($6 C$)}.}
		\item{$6 D$: Eine Ansicht der aktuellen Auswahl im Bildformat.}
		\item{$6 E$: Der Status zeigt die wichtigsten Daten über das ausgewählte Spielobjekt, hier \textit{Kosten} (K), \textit{Verteidigungsstärke} (VS), \textit{Angriffsintervall }(AI) und \textit{Reichweite} (RW)}
	\end{itemize}
	\item {$7$ Der Mauszeiger mit dem der Spieler die Auswahl trifft.}
\end{itemize}
Die anderen sichtbaren Objekte bilden die Spielwelt. Ein wichtiger Teil davon ist die Strecke die in Felder unterteilt ist. Eines dieser Felder belegt das Verteidigungsgebäude \textbf{Mauszeigerschütze (8)}, der in dieser Spielsituation gerade seinen Angriff durchführt und \textbf{Mauszeiger (10)} auf einen \textbf{Bug (12)} schießt.

Zwei weitere Verteidigungsgebäude sind ebenfalls zu sehen: einige \textbf{Kabel (11)} und der \textbf{CD-Werfer (9)} der gerade eine CD abfeuert.


\section{Kamera}
Wie bereits erwähnt handelt es sich bei der Kamera von \textit{Kernel Panic!} um eine \textit{Top-down Perspektive}, wie man es aus den typischen Tower Defense Spielen (z.B. Plants vs. Zombies) kennt.
Sie liefert eine strategische Übersicht auf die 2D Grafik und lässt sich mit der \textit{Maus steuern}.
Bewegt man den Mauszeiger an einen Bildschirmrand scrollt man in die entsprechende Richtung über die Spielwelt.

%\section{Tastaturzuweisung}



\section{Menü-Struktur}

% Siehe auch: https://sopra.informatik.uni-freiburg.de/soprawiki/Men%C3%BC

% Bei der Beschreibung der Menü-Struktur wird erklärt, wie das Hauptmenü und
% alle In-Game-Menüs zueinander in Beziehung stehen. Hilfreich dazu kann ein
% Diagramm in Form eines Graphen oder Baums sein.
%
% Wichtig ist, dass ersichtlich wird, welche Aktion im Menü welche Reaktion des
% Interfaces verursacht. Beispiel: "Wenn man im Einstellungsmenü auf 'Zurück'
% klickt, gelangt man zurück ins Hauptmenü."
%
% Ebenso wichtig ist die Vollständigkeit der Beschreibung. Jedes Menü und jedes
% Untermenü sollten erklärt werden.

% \missingSection{Menü-Struktur}
Beim Starten von \textit{Kernel Panic!} öffnet sich nach der Hintergrundgeschichte direkt das \textit{Hauptmenü} (siehe Abbildung \ref{fig:menu}). Hier hat man Zugriff auf die \textit{Spielanleitung}, \textit{Statistiken}, \textit{Achievements} und die \textit{Credits}.\\
Das wählen einer dieser Felder öffnet ein Fenster, in dem man diverse Informationen zum entsprechenden Thema einsehen kann. Mithilfe des Feldes \textit{Zurück} oder dem Betätigen der \textit{Escape}-Taste gelangt man wieder in das Hauptmenü.\\
Um das Spiel zu beenden wählt man im Hauptmenü das Feld \textit{Beenden}. Damit man nicht versehentlich das Spiel schließt öffnet sich zunächst noch ein zusätzliches Fenster, man kann nun entweder das Beenden bestätigen oder zurückkehren.\\
Das Feld \textit{Spielen} öffnet das Menü, dass für das Erstellen beziehungsweise das Laden des Spieles zuständig ist. Wenn man nicht durch Wählen der Escape-Taste oder Betätigen des \textit{Zurück}-Feldes das Hauptmenü öffnet, wählt man hier einen von fünf Spielständen, sogenannten \textit{Spielslots}. Jeder \textit{Spielslot} kann entweder genau einen zuvor gespeicherten Spielstand enthalten oder \textit{leer} sein.\\
Wenn man einen nicht-\textit{leeren} \textit{Spielslot} ausgewählt hat kann man ein angefangenes \textit{Spiel laden} oder ein \textit{Spiel erstellen}; falls der aktuelle \textit{Spielslot} \textit{leer} ist bleibt nur die Option ein neues \textit{Spiel} zu \textit{erstellen}.\\
Unabhängig davon ob der \textit{Spielslot} frei ist, gelangt man nun in das Spiel.\\
Während dem Spiel kann man zu jedem Zeitpunkt pausieren - es öffnet sich das \textit{Pause}-Menü.\\
Hier gibt es unter anderem die beiden Felder \textit{Speichern} und \textit{Laden}.\\
\textit{Speichern} ersetzt den gesicherten Spielstand des aktuellen \textit{Spielslots} durch eine Kopie des aktuellen Spiels zu diesem Zeitpunkt.\\
\textit{Spiel Laden} ersetzt das aktuell pausierte Spiel durch das zuvor gesicherte. Innerhalb des Spiels kann weder durch \textit{Spiel speichern}, noch \textit{Spiel laden} den \textit{Spielslot} wechseln, dafür müsste man zunächst in das \textit{Hauptmenü}, welches mithilfe des Feldes \textit{Hauptmenü} erreicht werden kann.\\
Es gibt noch ein weiteres Menü: die \textit{Optionen}, diese lassen sich durch wählen des Feldes \textit{Optionen} sowohl aus dem \textit{Hauptmenü} als auch direkt aus dem \textit{pausierten} Spiel öffnen. Dementsprechend führt auch das \textit{Zurück} Feld, beziehungsweise die \textit{Escape}-Taste wieder zum vorherigen Menü.\\
Es lassen sich hier nun verschiedene Audio-Einstellungen treffen. Soundeffekte und Musik können über je ein eigenes Feld stummgeschalten werden. Außerdem hat man die Möglichkeit über einen Schieberegler die Lautstärke der jeweiligen Komponente einzustellen.\\
Auch die Tastaturbelegung wird im Menü \textit{Optionen} angepasst. Es gibt für alle individualisierbare Aktionen die Möglichkeit die Standartbelegung zu ändern oder eine alternative Taste festzulegen.\\
Man wählt hierfür das zu ändernde Feld aus und überschreibt die gespeicherte Taste mit dem nächsten Input.

\begin{figure}[ht]
	\centering
	\includegraphics[width=1\textwidth]{menu_structure.png}
	\caption{Menü-Struktur}
	\label{fig:menu}
\end{figure}


% \includegraphics[width=\textwidth]{menu_structure.png}

\section{Tastaturbelegung}

Die Tastaturbelegung lässt sich im Optionsmenü
(Abschnitt~\ref{sec:menu-options}) anpassen. Standardmäßig ist nur die
\emph{Escape}-Taste belegt: ist der Spieler im Baumodus, wird dieser beendet;
ist der Spieler im Spiel, wird dieses pausiert, das Pausenmenü wird
eingeblendet (Abschnitt~\ref{sec:menu-pause}); ist der Spieler im Menü, kehrt
er mit \emph{Escape} in das vorhergehende Menü zurück.



\chapter{Technische Merkmale}

% Die technischen Merkmale beinhalten eine Übersicht über die unterschiedlichen
% Technologien, welche im Spiel verwendet werden.


\section{Verwendete Technologien}

% In diesem Abschnitt sollten alle verwendeten Technologien, die zur Erstellung
% des Spiels wichtig sind, stichpunktartig erwähnt werden. Das beinhaltet XNA
% ebenso wie eventuell verwendete externe Bibliotheken. Auch die Programme, die
% verwendet werden, um Modelle, Grafiken und Sounds zu erstellen werden hier
% erwähnt. Wenn zusätzliche Programme, wie zum Beispiel Physik Engines, vom
% Spiel vorausgesetzt werden, wird dies hier ebenso erwähnt.

\missingSection{Verwendete Technologien}


\section{Mindestvoraussetzungen}

% Dieser Abschnitt ist von der Form her vergleichbar mit den
% Mindestvoraussetzungen, die auf Spieleverpackungen gedruckt sind. Er
% beinhaltet die minimale Hardware, die notwendig ist, um das Spiel flüssig
% spielen zu können, sowie die benötigte Software und Bibliotheken. Um die
% Hardwarevoraussetzungen zu ermitteln, gibt es unterschiedliche Möglichkeiten:
%
%   • Die Hardwarespezifikationen des schlechtesten PCs eines Gruppenmitglieds,
%     auf dem das Spiel noch ohne Probleme läuft
%
%   • Die Ausstattung der Pool-Rechner (falls kein Gruppenmitglied einen PC hat,
%     auf dem das Spiel lauffähig ist)

\missingSection{Mindestvoraussetzungen}

\begin{itemize}[leftmargin=*, nosep]
    \item \dots
\end{itemize}



\chapter{Spiellogik}

% In diesem Abschnitt wird die gesamte Spielmechanik und alle im Spiel
% vorkommenden Spielobjekte erklärt. Sinn dieses Abschnittes ist es, eine
% Übersicht über alle Interaktionen des Spielers mit der Spielwelt und den
% Spielobjekten zu erhalten.

\todo[inline, caption = {Seitenumbrüche}]{%
  Wenn dieses Kapitels fertig ist, müssen evtl. Seitenumbrüche manuell
  eingefügt werden, um ein sinnvolles Ergebnis zu erzielen.}

\section{Spielobjekte}

% Hier sind alle Spielobjekte aufgelistet, die es im Spiel gibt. Dies können
% zum Beispiel Einheiten, Gebäude, Hindernisse usw. sein. Hilfreich zum
% Verständnis und zur Identifizierung der einzelnen Objekte können Bilder sein.
%
% Insbesondere erklärt dieser Abschnitt alle Eigenschaften und Fähigkeiten der
% unterschiedlichen Objekte. Sind sie zum Beispiel einfach zu zerstören, kosten
% sie viele Ressourcen, usw.
%
% Spielobjekte können hier auch schon mit konkreten Werten versehen werden;
% wenn noch keine konkreten Werte vorliegen (z.B. wie viele Lebenspunkte eine
% bestimmte Einheit hat), sollten zumindest die Verhältnisse der Werte von
% unterschiedlichen Spileobjekten aufgelistet werden, also zum Beispiel: Welche
% Einheit ist stärker als eine andere? Dass sich konkrete Werte noch verändern
% können, ist selbstverständlich, und eine Frage des Balancings gegen Ende des
% Projekts.

\missingSection{Spielobjekte}

\begin{table}[ht!]
  \caption{Eigenschaftswerte von \dots}
  \small
  \begin{longtabu}{cX[-1]X}
    \toprule\rowfont{\itshape}
    & Eigenschaft & Beschreibung \\
    \midrule

    % Example
    LP & Lebenspunkte & \dots \\

    \bottomrule
  \end{longtabu}
\end{table}


\begin{table}[ht!]
  \caption{Werte von Einheiten}
  \small
  \begin{longtabu}{cXl}
    \toprule

    \multicolumn{3}{c}{\bfseries Frosch} \\\midrule
    Beschreibung & \dots & \itshape Bild hier? \\
    LP & 1 \\
    \midrule[\heavyrulewidth]

    \multicolumn{3}{c}{\bfseries Dummy} \\\midrule
    Beschreibung & \dots & \itshape Bild hier? \\
    LP & 2 \\
    \bottomrule
  \end{longtabu}
\end{table}


\section{Optionen und Aktionen}

% Siehe auch: https://sopra.informatik.uni-freiburg.de/soprawiki/Game_Mechanic

% Dieser Abschnitt beinhaltet die Aktionen, die Spieler oder KI vornehmen
% können, um den Zustand des Spiels zu verändern (zB das Bauen von Einheiten
% oder das Abbauen von Ressourcen). Je klarer diese Aktionen formuliert sind,
% desto leichter fällt einem die Umsetzung der Aktionen bei der Programmierung
% des Spiels.
%
% Wichtig sind auch die Einstellungen, die der Spieler am Spiel vornehmen kann,
% um das Spielverhalten zu verändern (zB.  Schwierigkeitsgrad ändern).
%
% Das Ziel des Spiels sollte schließlich anhand der beschriebenen Aktionen
% erklärt werden.
%
% Die Auflistung der Optionen und Aktionen erfolgt tabellarisch und ist in Form
% und Inhalt an Use Cases (http://de.wikipedia.org/wiki/Use_case) angelehnt.

\missingSection{Optionen und Aktionen}

\StartId{A}
\begingroup
  \small
  \tabulinesep=1mm
\begin{longtabu}{X[0.6L]X[0.4L]X[L]X[L]X[L]}
  \rowfont{\normalsize}
  \caption{Mögliche Optionen und Aktionen\label{tab:optionen-aktionen}}\\
  \midrule[\heavyrulewidth]\rowfont{\itshape}
    ID/Name              &
    Akteure              &
    Ereignis"-fluss      &
    Anfangs"-bedingung   &
    Abschluss"-bedingung \\
  \midrule\endfirsthead

  \rowfont{\normalsize}
  \caption[]{Mögliche Optionen und Aktionen (fortges.)}\\
  \midrule[\heavyrulewidth]\rowfont{\itshape}
    ID/Name              &
    Akteure              &
    Ereignis"-fluss      &
    Anfangs"-bedingung   &
    Abschluss"-bedingung \\
  \midrule\endhead

  \multicolumn{5}{r}{\itshape fortges. auf der nächsten Seite}\\
  \endfoot

  \endlastfoot
  
  \defid[object-selection]{Objekt Auswahl}
  & Spieler
  & \begin{enumerate}[nosep, leftmargin=*]
  \item Spieler clickt auf das Objekt (Turm oder Held).
  \item Objekt ist jetzt ausgewählt.
  \end{enumerate}
  & Objekt ist auf einer der 2 Strecken und gehört dem Spieler.
  & Objekt ist ausgewählt.
  
    \\\midrule
    \defid[cancel-object-selection]{Objekt Auswahl kündigen}
    & Spieler
    & \begin{enumerate}[nosep,leftmargin=*]
    	\item Spieler druckt TASTE, was den Objekt-Auswahl terminiert.
    \end{enumerate}
    & Ein Objekt (Turm oder Held) ist gerade ausgewählt.
    & Objekt nicht mehr ausgewählt.
	\\\midrule
  % Example line taken from the SOPRA wiki.
  \defid[move-figure]{Figur(en) durch Klick bewegen}
    % This can now be referenced using \refid{A:move-figure}
    & Spieler
    & \begin{enumerate}[nosep,leftmargin=*]
    	\item Click innerhalb der Angriffstrecke.
    	\item Figur(en) bewegen sich zur
    	angegebenen Position.
    \end{enumerate}
    & Der Spieler muss eine oder mehr kontrollierbare, auswählbare Spielfiguren
      ausgewählt haben.
    & Die Spielfiguren befinden sich am Zielpunkt, \textbf{oder} die
      Spielfiguren befinden an einem begehbaren Punkt in der Welt, der möglichst
      nah am Zielpunkt liegt.
      
  \\\midrule
  \defid[hero-ability]{Held Fähigkeit Auswahl}
  & Spieler
  & \begin{enumerate}[nosep, leftmargin=*]
  \item click auf Fähigkeit.
  \item Führe Fähigkeit aus
  \end{enumerate}
  & Held ist ausgewählt und kein aktiver cool-down timer für die Fähigkeit.
  & Fähigkeit war ausgeführt.      
  
  \\\midrule
  \defid[select-building]{Gebäude aus Menu auswählen}
  & Spieler
  & \begin{enumerate}[nosep, leftmargin=*]
  \item Gebäude wird durch click ausgewählt
  \end{enumerate}
  & Maus hängt über einen Gebäude-Menueintrag
  & Gebäude ist jetzt ausgewählt und man kann es auf der Verteidigungsstrecke herum ziehen um
  Baustelle auszuwählen.
  
  \\\midrule
  \defid[put-defend]{Gebäude kaufen}
  & Spieler
  &  \begin{enumerate}[nosep, leftmargin=*]
  \item Spieler clickt auf der Verteidigungsstrecke.
  \item Geld für das Gebäude wird abgezogen.
  \item Gebäude wird auf die Ausgewählte Stelle plaziert.
  \end{enumerate}
  & Genug Gold vorhanden,Maus auf der Verteidigungsstrecke und 
  Umgebung nicht belegt.
  & Gebäude wird auf der Position gebaut
  
  \\\midrule
  \defid[buy-attack]{Einheit kaufen.}
  & Spieler
  & \begin{enumerate}[nosep,leftmargin=*]
    \item Spieler clickt.
    \item Gold wird abgezogen.
    \item Einheit wird produziert
    \end{enumerate}
  & \begin{enumerate}[nosep,leftmargin=*]
  \item Mauszeiger zeigt auf eine Einheitskarte im Einheitenmenu.
  \item Genug Gold vorhanden.
  \end{enumerate}
  & Einheit produziert.
  
  \\\midrule
  \defid[update-defend]{Turm updaten}
  & Spieler
  &\begin{enumerate}[nosep, leftmargin=*]
  \item Spieler clickt auf Update.
  \item Turm ist aktualisiert.
  \end{enumerate}
  & Turm war vorher ausgewählt und genug Gold vorhanden.
  & Turm ist up-to-date.
  
    \\\midrule
    \defid[select-Upgrade]{Upgrade auswählen}
    & Spieler
    & \begin{enumerate}[nosep,leftmargin=*]
    	\item click
    	\item Wenn genug XP, Fähigkeit wird eingeschaltet. Sonst, irgendeine Meldung.
    \end{enumerate}
    & Mauszeiger befindet sich über einen Upgrade-Node
    & Upgrade wird angewendet oder Meldung wird angezeigt.  
  
  \\\midrule
  \defid[unit-hit]{Turm wird angegriffen}
  & Turm
  & \begin{enumerate}[nosep, leftmargin=*]
  \item Lebenspunkte werden abgezogen
  \begin{itemize}[nosep, leftmargin=*]
  	\item Wenn Lebenspunkte noch positiv, das war es
  	\item Sonst Turm wird vollständing zerstört
  \end{itemize}
  \end{enumerate}
  & gegnerische Angriffseinheit, greift Turm aktiv an.
  & Turm hat weniger Lebenspunkte, oder Turm ist zerstört.
  
    \\\midrule
    \defid[tower-attack]{Turm Angriff}
    & Turm
    &\begin{enumerate}[nosep, leftmargin=*]
    \item Turm greift Gegner an solange Gegner nah genug.
    \end{enumerate}
    & Gegner in Reichweite vom Turm.
    & Gegner getötet oder entkommen.

    \\\midrule
    \defid[die]{Angriffseinheit stirbt}
    & Einheit
    & \begin{enumerate}[nosep,leftmargin=*]
    \item Einheit stribt (nicht mehr auf dem Spielfeld)
    \end{enumerate}
    & Einheit hat 0 Lebenspunkte.
    & Einheit nicht mehr sichtbar und nicht mehr verfügbar.
    
    \\\midrule
    \defid[damage-base]{Gegner-Basis Angreifen}
    &Angriffs"-einheit.
    & \begin{enumerate}[nosep, leftmargin=*]
    \item Entsprechend viele Lebenspunkte sind von der Gegner-Basis abgezogen
    \item Angriffseinheit wird gelöscht.
    \end{enumerate}
    & Angrffseinheit hat Gegner-Basis erreicht.
    & Gegner-Basis hat weniger Lebenspunkte, Angriffseinheit nicht mehr verfügbar.
    
    \\\midrule
    \defid[base-die]{Basis kaputt}
    & Basis.
    & \begin{enumerate}[nosep, leftmargin=*]
    \item Lebenspunkten sind abgezogen.
    \item \begin{itemize}[nosep, leftmargin=*]
    	\item wenn Lebenspunkte noch positiv, das war es.
    	\item Sonst Basis kaputt, Spiel vorbei.
    \end{itemize}
    \end{enumerate}
    & Gegnerische Angriffseinheit hat Basis erreicht.
    & Basis jetzt mit weniger Lebenspunkte oder kaputt.
  \\

  \bottomrule
\end{longtabu}
\endgroup


\section{Spielstruktur}

% Dieser Abschnitt erklärt den Ablauf des Spiels. Das heißt, hier wird
% beschreiben, was geschieht, sobald der Spieler ein neues Spiel beginnt und
% wie sich das Spiel von dort aus entwickelt, bis es gewonnen oder verloren
% ist. Eine Beschreibung der unterschiedlichen Spielphasen ist hier essentiell.
%
% Eine mögliche Einteilung der Spielphasen von Schach ist zum Beispiel:
% Early-Game (Eröffnung), Mid-Game (strategische Positionen festigen),
% Late-Game (wenn nur noch wenige Figuren auf dem Brett sind).
%
% Eine weitere wichtige Information in diesem Abschnitt ist, welche Modi das
% Spiel hat (zum Beispiel Missionen und wie sie sich unterscheiden vs.
% Endlosmodus und wie dieser während des Spiels verändert wird).
%
% Außerdem soll die Dynamik des Spiels beschrieben werden (Beispiel: statisch,
% d.h. wenig Veränderungen an der Spielwelt und den Mechaniken vs. actionreich
% und dynamisch).
%
% Auch mögliche Taktiken und Strategien im Spiel können hier beschrieben
% werden.

Das Spiel ist ein dynamisches Spiel. 

\subsection{Kaufoptionen}
\label{subsec:kaufoptionen}
    Alle Angriffs- und Verteidigungseinheiten, deren Verbesserungen und deren
    Upgrades können zu jedem Zeitpunkt während des Spiels gekauft werden, 
    solange genug Gold vorhanden ist. Unterschiede bestehen beim 
    Aktivierungszeitpunkt.

	\begin{enumerate}
		\item Verbesserungen: Die Veränderung tritt sofort in Kraft.
		\item Upgrades: Die Veränderung tritt sofort in Kraft.
    		\item Kaufen neuer Einheiten
		\begin{enumerate}
			\item Türme: Sobald der gewählte Platz leer ist wird
				gebaut.
			\item Angriffseinheiten: Käufe bildet die Angreifer 
				für die nächste Welle.
    		\end{enumerate}
	\end{enumerate}

\subsection{Spielablauf}
Beim Starten eines neuen Spiels erscheint ein Feld mit der
Hintergrundgeschichte. Hier wird kurz in die Idee hinter der Spielwelt
eingeführt. //
Wenn das Feld geschlossen landet man bei der ersten Spielwelt. Jetzt hat der 
Spieler Zeit die erste Welle vorzubereiten, sprich seine Türme aufzustellen 
und seine Angriffseinheiten zu kaufen. Dafür hat der Spieler 50 Einheiten Gold
zur Verfügung. //
Nun kann die erste Welle gestartet werden.
Nun laufen die Angriffseinheiten in kurzen Abständen hintereinander von der
eigenen Basis los, auf die gegnerischen Türme zu. Die Reihenfolge entspricht
der Kaufreihenfolge. Zusätzlich tauchen die drei kontrollierbaren,
kollidierenden und bewegbaren Einheiten (Helden) am Ausgang der Basis auf. Diese
können von nun an mit der Maus gesteuert werden. \\
Sobald das Spiel gestartet wird verdient der Spieler über Zeit Gold. Dieses
kann auch sofort wieder investiert werden. Für jedes Goldstück besteht die
Wahl zwischen Angriff und Verteidigung(\ref{subsec:kaufoptionen}). \\

Die nächste Welle wird gestartet sobald einer der beiden Spieler keine 
Angriffseinheiten, mit Ausnahme der Helden, mehr auf dem Spielfeld besitzt
und beide Basen noch Leben besitzen. Die verbliebenen Einheiten des zweiten 
Spielers attakieren weiter bis sie ebenfalls tot sind. \\
Sobald die gesamte Angriffseinheit, mit Ausnahme der Helden, tot ist oder die 
gegnerische Basis erreicht hat, bekommt dieser Spieler einen Erfahrungspunkt. 
Außerdem bekommt der Spieler mit der höheren Anzahl an Angreifern die 
Differenz der Anzahl der Angreifer der beiden Spieler in Gold gut geschrieben.
Die Höhe dieses Betrags ist jedoch pro Welle limitiert und das Limit steigt 
mit zunehmender Anzahl an Wellen an.\\

Dieser Prozess wiederholt sich solange bis einer der beiden Basen zerstört 
wurde. Der Spieler, der die Basis zerstört hat, gewinnt das Spiel. 
Jetzt erscheint ein Gewonnen oder Verloren Bildschirm vom dem in das Hauptmenü
zurück gekehrt werden kann.


\section{Statistiken}

% Statistiken erlauben es einem Spieler, sich mit anderen Spielern zu messen
% und zu entscheiden, wer besser ist und sind ein wichtiger Bestandteil von
% nahezu jedem Spiel.
%
% In diesem Abschnitt wird erklärt, welche unterschiedlichen Statistiken
% während des Spiels gesammelt werden, wie sie Einfluss auf das Spielgeschehen
% geben und wodurch die unterschiedlichen Werte während des Spiels geändert
% werden.
%
% Das einfachste Beispiel für Statistiken sind Highscore-Listen, in denen die
% größte erreichte Punktzahl eines Spieldurchlaufs pro Spieler aufgelistet ist.

Kernel Panic! sammelt für jeden Spieldurchlauf die folgenen Statistiken:
\begin{itemize}
	\item Sieger-Seite
	\item Dauer der Spielzeit
	\item APM (actions per minute)
	\item Anzahl besiegter gegnerischer Einheiten / Total Damage dealt
	\item Bitcoin investiert in Angriffseinheiten
	\item Bitcoin investiert in Verteidigungsgebäude
	\item Bitcoin investiert in Upgrades
	\item Bitcoin investiert in Special-Upgrades
	\item Average Bitcoin Leak (Wie viel Bitcoin hat der Gegner zusätzlich, durch Überqueren des eigenen Territoriums, im Schnitt pro Minute erbeutet)
	\item Average Bitcoin Bonus (vice versa)
\end{itemize}

%\missingSection{Statistiken}


\pagebreak
\section{Achievements}

% In diesem Abschnitt werden die unterschiedlichen Achievements und die
% Bedingungen, wie diese erreicht werden können, aufgelistet.
%
% Achievements können verschiedene Schwierigkeitsstufen besitzen, um sie zu
% erreichen. Achievements mit unterschiedlichen Schwierigkeitsstufen sind z.B.:
% "Zerstöre 2000 gegnerische Einheiten" vs. "Schaffe das gesamte Spiel, ohne
% einen Schuss abzufeuern".

Tabelle~\ref{tab:achievements} beschreibt die Achievements in \emph{Kernel Panic!.}
\begingroup
  \small
  \tabulinesep=1.2mm
  \begin{longtabu}{X[0.7]X}
    \rowfont{\normalsize}
    \caption{Mögliche Achievements in \emph{Kernel Panic!}\label{tab:achievements}}\\
    \midrule[\heavyrulewidth]\rowfont{\itshape}
    Name & Beschreibung \\
    \midrule\endfirsthead

    \rowfont{\normalsize}
    \caption[]{Achievements (fortges.)}\\
    \midrule[\heavyrulewidth]\rowfont{\itshape}
    Name & Beschreibung \\
    \midrule\endhead

    \bottomrule
    \multicolumn{2}{r}{\emph{fortges. auf der nächsten Seite}} \\
    \endfoot

    \endlastfoot

    % Example
    First Victory! / GG EASY / Is Dis Tetris?
      & Du hast das Spiel zum 1./10./100. Mal gewonnen!
      	\\
    Unlucky Loss / Rekt / Complete Humiliation
    	& Du hast das Spiel zum 1./10./100. Mal verloren
    	\\
	Minion Slayer
		& Du hast in einem Spiel X Angriffseinheiten getötet
		\\
	Bitcoin Addiction
		& Du besitzt in einem Spiel über X Bitcoin
		\\
	while true DO sudo apt-get upgrade DONE
		& Du hast in einem Spiel über X Bitcoin in Upgrades investiert
		\\
	Iron Fortress
		& Du gewinnst das Spiel mit einer Ladung von 100\%
		\\
	Tower's win the game
		& Du hast in einem Spiel über X Bitcoin in Verteidigungsgebäude investiert
		\\
	Bitcoin Thief
		& Du hast in einem Spiel über X Bitcoin zusätzlich durch Überqueren des gegnerischen
		Territoriums erbeutet
		\\
	Bank Account Hacked!
		& Dein Gegner hat in einem Spiel über X Bitcoin zusätzlich durch Überqueren deines
		Territoriums erbeutet
		\\
	Dirty Coder
		& Du hast in einem Spiel über X Bug-Einheiten gebaut
		\\
	Fix your Code!
		& Du hast in einem Spiel über X Bug-Einheiten besiegt
		\\
	Bzzzz
		& Du hast in einem Spiel X Einheiten mit Schockfeld besiegt
		\\
	APM God
		& Du hast in einem Spiel über X APM
		\\
	Idle Gamer
		& Du hast in einem Spiel über X Runden nichts gebaut/abgerissen oder Bitcoin in upgrades investiert
		\\
	Hacker
		& Du hast in einem Spiel über X Virus-/Trojaner-Einheiten gebaut
		\\
	High Security Anti-Virus
		& Du hast in einem Spiel über X Virus-/Trojaner-Einheiten besiegt
		\\
	\st{Tower Defense} Jump 'n' Run
		& Du hast in einem Spiel X Verteidigungsgebäude mit der Firefox-Einheit übersprungen
		\\
	Fool!
		& Du hast versucht einen leeren Spielstand zu laden.
		\\
	Nutcracker
		& Du hast eine Nokia-Einheite besiegt.
		\\
	Teure Leitung
		& Du hast X viele Kabel-Einheiten gebaut
		\\
	High Inference
		& Gewinne nur mit Wifi-Router als Verteidigungsgebäude
		\\
    \bottomrule
  \end{longtabu}
\endgroup



\chapter{Screenplay}

% Dieser Abschnitt beinhaltet die Hintergrundgeschichte (Story) des Spiels.
% Eine Story ist wichtig für ein Spiel, um zu erklären, warum bestimmte
% Aktionen durchgeführt werden können, oder nicht. Eine Story ist auch ein
% einfacher Weg, um eine Umgebung zu schaffen, mit der sich ein Spieler
% identifizieren kann und Spaß daran hat, die Umgebung zu erforschen und sich
% darin zu bewegen.

Im Jahr 1987 entwickelst du einen revolutionären Supercomputer.
Doch mittlerweile interessiert sich kaum noch jemand für deine Erfindung.
Zu allem Überfluss wirst du jetzt auch noch von einem Hacker angegriffen der sich zum Ziel gesetzt hat dein Lebenswerk zu zerstören.
Deine einzige Unterstützung ist ein russisches Hacker-Kollektiv, das dich mit Viren für den Gegenangriff versorgt.

\section{Konzeptzeichnungen und Storyboards}

% Bilder sind wichtig für den ersten Eindruck. Vor allem im GDD sind
% Konzeptzeichnungen und Skizzen gut aufgehoben. Auf diese Weise kann man nicht
% nur sich selbst schnell eine Vorstellung von den Ideen machen, sondern auch
% anderen vermitteln, worum es im Spiel geht und wie das Spiel und seine
% Geschichte aussieht.

\missingSection{Konzeptzeichnungen und Storyboards}
\begin{figure}[ht]
	\centering
	\includegraphics[]{pferd.png}
	\caption{Trojaner}
	\label{fig:pferd}
\end{figure}


\end{document}
