\subsection[Verteidigungsgebäude]{Verteidigungsgebäude \hyperref[sec:defence]{\footnotesize $\rightarrow$ Spielobjekte}}
\label{sec:defence-concept}

Der Spieler baut Gebäude, um die gegnerischen Einheiten auf ihrem Weg zur Basis
aufzuhalten und zu töten, damit die Einheiten nicht erfolgreich Schaden
verursachen können. Jedes Gebäude nimmt eine Kachel vollständig ein, sodass
weder Angriffseinheiten darüber laufen, noch weitere Gebäude auf dieses Feld
platziert werden können.

Um ein Gebäude zu errichten muss der Spieler Bitcoin zahlen. Der
\emph{Gebäudewert,} aus dem sich in machen Fällen weiterführende Kosten
berechnen, ist die Summe an Bitcoins, die in dieses Gebäude investiert wurden.


\subsubsection{Gebäudearten}

Wie die Angriffseinheiten lassen sich auf die Gebäude in Unterkategorien
aufteilen: Es gibt aktive Gebäude -- \emph{Türme} genannt --, die feindlichen
Einheiten Schaden zufügen oder einen temporären Malus auferlegen, und passive
Gebäude, die lediglich Wege versperren und Einheiten damit zwingen, längere
Routen zu wählen.

Türme haben einen \emph{Radius.} Nur Einheiten innerhalb dieses Radius werden
anvisiert und vom Effekt des Turms (wie beispielsweise „Schaden zufügen“)
getroffen. Nachdem ein Turm aktiv geworden ist, muss zuerst die
\emph{Abklingzeit} vergehen, bevor er erneut aktiv werden kan. Ein neu gebauter
Turm kann direkt aktiv werden.

Manche Türme verschießen gezielt einzelne Projektile, andere Türme haben einen
Flächeneffekt oder schießen ungezielt. Projektile, die gezielt verschossen
werden, haben keine Treffergarantie; vielmehr fliegen sie gradlinig in die
Richtung, in die der Turm beim Abschuss gezielt hat.


\subsubsection{Aktionen auf Gebäuden}

Ist ein eigenes Gebäude ausgewählt, so kann man Aktionen auf diesem Gebäude
ausführen. Je nach Gebäudetyp steht eventuell nur eine Untermenge der im
Folgenden beschriebenen Aktionen zur Verfügung.

\begin{description}
  \item[Verkaufen] Gebaute Gebäude lassen sich jederzeit wieder verkaufen. Der
    Spieler erhält 80\,\% des Gebäudewertes zurück, das Feld ist für
    Einheiten wieder begehbar und kann mit anderen Gebäuden bebaut werden.

  \item[Verbessern] Türme können bis zu zweimal verbessert werden. Dies erhöht
    ihre Verteidigungsstärke und ihren Gebäudewert. Eine Verbesserung kostet
    den Spieler 25\,\% des aktuellen Gebäudewertes.

  \item[Strategie wählen] Bei gezielt schießenden Türmen ist es möglich,
    auszuwählen, welche Einheit im Radius anvisiert wird. Die Optionen sind:
    \begin{description}[itemsep=0pt]
      \item[Erste Einheit] \emph{(ist standardmäßig ausgewählt)}\\
        Greift die Einheit an, die den kürzesten Weg hat, um Schaden an der Basis
        zu verusachen.

      \item[Stärkste Einheit] ~\\
        Greift die Einheit an, die die meisten LP hat.

      \item[Schwächste Einheit] ~\\
        Greift die Einheit an, die die wenigsten LP hat.
    \end{description}

\end{description}
