\section{Statistiken}

% Statistiken erlauben es einem Spieler, sich mit anderen Spielern zu messen
% und zu entscheiden, wer besser ist und sind ein wichtiger Bestandteil von
% nahezu jedem Spiel.
%
% In diesem Abschnitt wird erklärt, welche unterschiedlichen Statistiken
% während des Spiels gesammelt werden, wie sie Einfluss auf das Spielgeschehen
% geben und wodurch die unterschiedlichen Werte während des Spiels geändert
% werden.
%
% Das einfachste Beispiel für Statistiken sind Highscore-Listen, in denen die
% größte erreichte Punktzahl eines Spieldurchlaufs pro Spieler aufgelistet ist.

\emph{Kernel Panic!} sammelt für jeden Spieldurchlauf die folgenen Statistiken:

\begin{itemize}[noitemsep, leftmargin=*]
	\item Sieger-Seite
	\item Dauer der Spielzeit
	\item APM (actions per minute)
	\item Anzahl besiegter gegnerischer Einheiten / Total Damage dealt
	\item Bitcoin investiert in Angriffseinheiten
	\item Bitcoin investiert in Verteidigungsgebäude
	\item Bitcoin investiert in Upgrades
	\item Bitcoin investiert in Special-Upgrades
	\item Average Bitcoin Leak (Wie viel Bitcoin hat der Gegner zusätzlich, durch Überqueren des eigenen Territoriums, im Schnitt pro Minute erbeutet)
	\item Average Bitcoin Bonus (vice versa)
\end{itemize}

%\missingSection{Statistiken}
