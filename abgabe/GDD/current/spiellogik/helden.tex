\section{Helden}

% Siehe auch: https://sopra.informatik.uni-freiburg.de/soprawiki/Game_Mechanic

% Dieser Abschnitt beinhaltet die Aktionen, die Spieler oder KI vornehmen
% können, um den Zustand des Spiels zu verändern (zB das Bauen von Einheiten
% oder das Abbauen von Ressourcen). Je klarer diese Aktionen formuliert sind,
% desto leichter fällt einem die Umsetzung der Aktionen bei der Programmierung
% des Spiels.
%
% Wichtig sind auch die Einstellungen, die der Spieler am Spiel vornehmen kann,
% um das Spielverhalten zu verändern (zB.  Schwierigkeitsgrad ändern).
%
% Das Ziel des Spiels sollte schließlich anhand der beschriebenen Aktionen
% erklärt werden.
%
% Die Auflistung der Optionen und Aktionen erfolgt tabellarisch und ist in Form
% und Inhalt an Use Cases (http://de.wikipedia.org/wiki/Use_case) angelehnt.

\missingSection{Helden in Tabelle}

\StartId{A}
\begingroup
  \small
  \tabulinesep=1mm
\begin{longtabu}{X[0.6L]X[0.4L]X[L]X[L]X[L]}
  \rowfont{\normalsize}
  \caption{Helden\label{tab:helden}}\\
  \midrule[\heavyrulewidth]\rowfont{\itshape}
    ID/Name              &
    Akteure              &
    Ereignis"-fluss      &
    Anfangs"-bedingung   &
    Abschluss"-bedingung \\
  \midrule\endfirsthead

  \rowfont{\normalsize}
  \caption[]{Helden (fortges.)}\\
  \midrule[\heavyrulewidth]\rowfont{\itshape}
    ID/Name              &
    Akteure              &
    Ereignis"-fluss      &
    Anfangs"-bedingung   &
    Abschluss"-bedingung \\
  \midrule\endhead

  \multicolumn{5}{r}{\itshape fortges. auf der nächsten Seite}\\
  \endfoot

  \endlastfoot

  % Example line taken from the SOPRA wiki.
  \defid[settings]{Ubuntu-settings}
    % This can now be referenced using \refid{A:move-figure}
    & Spieler
    & Spieler kauft für 10 Gold einen Firefox. Dieser wird per Mauszeiger gesteuert und heilt eigene Truppen, die sich in einem gewissen Radius befinden.
    & Spieler kauft Ubuntu-settings, welcher in der eigenen Basis spawned.
    & Ubuntu-settings wurde bei Angriff von gegnerischen Türmen getötet, \textbf{oder} ist bis zur gegnerischen Basis durchgedrungen, wo er jetzt Schaden anrichtet.
  \\\midrule
  \defid{Firefox}
    & Spieler
    & Spieler kauft für 10 Gold einen Firefox. Dieser wird per Mauszeiger gesteuert und kann gegnerische Verteidigungsgebäude überspringen.
    & Spieler kauft Firefox, welcher in der eigenen Basis spawned.
    & Firefox wurde von gegnerischen Türmen getötet, \textbf{oder} ist bis zur gegnerischen Basis durchgedrungen, wo er jetzt Schaden anrichtet.
  \\\midrule
  \defid{Bluescreen}
    & Spieler
    & Spieler kauft für 10 Gold einen Bluescreen. Dieser wird per Mauszeiger gesteuert und kann gegnerische Verteidigungsgebäude in der Nähe für kurze Zeit deaktivieren.
    & Spieler kauft Bluescreen, welcher in der eigenen Basis spawned.
    & Bluescreen wurde von gegnerischen Türmen getötet, \textbf{oder} ist bis zur gegnerischen Basis durchgedrungen, wo er jetzt Schaden anrichtet.
   \\\midrule

  \bottomrule
\end{longtabu}
\endgroup
