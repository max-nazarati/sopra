\section{Optionen und Aktionen}

% Siehe auch: https://sopra.informatik.uni-freiburg.de/soprawiki/Game_Mechanic

% Dieser Abschnitt beinhaltet die Aktionen, die Spieler oder KI vornehmen
% können, um den Zustand des Spiels zu verändern (zB das Bauen von Einheiten
% oder das Abbauen von Ressourcen). Je klarer diese Aktionen formuliert sind,
% desto leichter fällt einem die Umsetzung der Aktionen bei der Programmierung
% des Spiels.
%
% Wichtig sind auch die Einstellungen, die der Spieler am Spiel vornehmen kann,
% um das Spielverhalten zu verändern (zB.  Schwierigkeitsgrad ändern).
%
% Das Ziel des Spiels sollte schließlich anhand der beschriebenen Aktionen
% erklärt werden.
%
% Die Auflistung der Optionen und Aktionen erfolgt tabellarisch und ist in Form
% und Inhalt an Use Cases (http://de.wikipedia.org/wiki/Use_case) angelehnt.

In Tabelle~\ref{tab:optionen-aktionen} sind die in \emph{Kernel Panic!}
möglichen Optionen und Aktionen beschriebenen. Tabelle~\ref{tab:helden} geht
auf die in (\refid{A:hero-ability}) referenzierten Heldenfähigkeiten genauer
ein. Die Aktionen auf Gebäuden sind in Tabelle~\ref{tab:building-actions}
beschrieben. In Tabelle~\ref{tab:build-mode} sind die Optionen und Aktionen des
Baumodus, der mit (\refid{A:select-building}) aktiviert wird, erläutert.

\StartId{A}
\begingroup
  \small
  \tabulinesep=1mm
\begin{longtabu}{X[0.6L]X[0.4L]X[L]X[L]X[L]}
  \rowfont{\normalsize}
  \caption{Mögliche Optionen und Aktionen in \emph{Kernel Panic!}.\label{tab:optionen-aktionen}}\\
  \midrule[\heavyrulewidth]\rowfont{\itshape}
    ID/Name              &
    Akteure              &
    Ereignis"-fluss      &
    Anfangs"-bedingung   &
    Abschluss"-bedingung \\
  \midrule\endfirsthead

  \rowfont{\normalsize}
  \caption[]{Mögliche Optionen und Aktionen (fortges.)}\\
  \midrule[\heavyrulewidth]\rowfont{\itshape}
    ID/Name              &
    Akteure              &
    Ereignis"-fluss      &
    Anfangs"-bedingung   &
    Abschluss"-bedingung \\
  \midrule\endhead

  \multicolumn{5}{r}{\itshape fortges. auf der nächsten Seite}\\
  \endfoot

  \endlastfoot

  \defid[object-selection]{Objekt auswählen}
    & Spieler
    & Spieler klickt mit der Linken Maustaste auf einen Turm oder eine Einheit.
    & % Keine Voraussetzung
    & Turm oder Einheit ist ausgewählt. Ist es ein eigenes Gebäude kann eine
      Aktion aus Tabelle~\ref{tab:building-actions} ausgeführt werden.
  \\\midrule

  \defid[cancel-object-selection]{Objektauswahl kündigen}
    & Spieler
    & Spieler klickt mit der Linken Maustaste auf einen Bereich des
      Spielfeldes, der keine andere Aktion auslöst.
    & Türme oder Einheiten sind ausgewählt.
    & Türme und Einheiten sind nicht mehr ausgewählt.
	\\\midrule

  \defid[hero-ability-indicate]{Heldenfähigkeit indizieren}
	& Spieler
  & Der Spieler klickt mit der linken Maustaste auf den Fähigkeiten-Button. Darauf folgt entweder \refid{A:hero-ability} oder \refid{A:hero-ability-abort}.
	& Ein eigener Held ist ausgewählt und seine Fähigkeit ist bereit.
	& Ein Held ist ausgewählt, seine Fähigkeit ist bereit und der Einflussbereich der Fähigkeit wird angezeigt.
  \\\midrule

  \defid[hero-ability]{Heldenfähigkeit Indizierung bestätigen}
	& Spieler
	& Der Spieler bestätigt die Bereichsauswahl der Heldenfähigkeit mit der mittleren Maustaste.
	& Ein Held ist ausgewählt und seine Fähigkeit wird indiziert.
	& Fähigkeit wird ausgeführt, siehe Tabelle~\ref{tab:helden}.
	\\\midrule

  \defid[hero-ability-abort]{Heldenfähigkeit Indizierung abbrechen}
	& Spieler
	& Der Spieler bricht die Bereichsauswahl der Heldenfähigkeit mit der rechten Maustaste ab.
	& Ein Held ist ausgewählt und seine Fähigkeit wird indiziert.
	& Der Held ist ausgewählt und seine Fähigkeit ist bereit, die Indizierung der Fähigkeit ist abgebrochen.
	\\\midrule

  \defid[move-figure]{Helden durch Klick bewegen}
    & Spieler
    & \vspace*{-0.2cm}\begin{enumerate}[nosep,leftmargin=*]
        \item Spieler klickt mit der Rechten Maustaste auf einen Punkt auf der
          Angriffsstrecke.
        \item Die ausgewählten eigenen kontrollierbaren Einheiten bewegen sich
          auf dem kürzesten Weg auf den ausgewählten Punkt zu, Hindernisse
          werden rechtzeitig umlaufen.
      \end{enumerate}
    & Eine eigener Held ist ausgewählt, das Spiel befindet sich nicht in der
      Vorbereitungszeit.
    & Die ausgewählten eigenen kontrollierbaren Spielfiguren befinden sich am
      Zielpunkt, \textbf{\textls{oder}} die befinden sich an dem erreichbaren
      Punkt, der möglichst nah am Zielpunkt liegt.
  \\\midrule

  \defid[select-upgrade]{Upgrade auswählen}
    & Spieler
    & Der Spieler klickt mit der Linken Maustaste auf ein Upgrade, das
      angewendet wird.
    & Es sind genügend Erfahrungspunkte verfügbar.
    & Upgrade wurde angewendet.
  \\\midrule

  \defid[tower-attack]{Turm verteidigt}
    & Turm
    & Der Turm führt seinen Angriff durch.
    & Einheit befindet sich in der Reichweite des Gebäudes und das Gebäude ist
      bereit zum Angriff.
    & Turm muss Angriffsintervall abwarten um den nächsten Angriff durchführen
      zu können; Einheiten werden getroffen (\refid{A:unit-hit}).
  \\\midrule

  \defid[unit-hit]{Angriffseinheit erhält Schaden}
    & Angriffs"-einheit
    & Die Angriffseinheit verliert Leben entsprechend der Verteidigungsstärke
      des Turms.
    & Einheit wurde von einem Turm angegriffen (\refid{A:tower-attack}).
    & Angriffseinheit hat weniger Lebenspunkte.
  \\\midrule

  \defid[die]{Angriffseinheit stirbt}
    & Angriffs"-einheit
    & \vspace*{-0.2cm}\begin{enumerate}[nosep,leftmargin=*]
        \item Einheit stirbt.
        \item Wenn die Einheit \emph{Trojaner} ist, wird
          \refid{A:spawn-children} ausgeführt.
      \end{enumerate}
    & Angriffseinheit hat 0 oder weniger Lebenspunkte.
    & Angriffseinheit nicht mehr sichtbar und nicht mehr verfügbar.
  \\\midrule

  \defid[spawn-children]{Trojaner spawnt Einheiten}
    & \emph{Tro"-janer}
    & An der Stelle an der die Eineit gestorben ist, werden fünf \emph{Bugs}
      gespawnt. Diese werden der gleichen Welle zugeordnet wie der Trojaner.
    & Trojaner ist gestorben.
    & \emph{Bugs} und \emph{Viren} sind gespawnt worden.
  \\\midrule

  \defid[damage-base]{Gegnerische Basis angreifen}
    & Angriffs"-einheit.
    & \vspace*{-0.2cm}\begin{enumerate}[nosep, leftmargin=*]
        \item Die Basis verliert Ladung entsprechend der Angriffsstärke der
          Einheit.
        \item Angriffseinheit wird gelöscht.
      \end{enumerate}
    & Angrffseinheit hat die gegnerische Basis erreicht.
    & Gegnerische Basis hat weniger Ladung, die Angriffseinheit ist nicht mehr
      verfügbar.
  \\\midrule

  \defid[base-die]{Basis stirbt}
    & Basis
    & Das Spiel ist beendet.
    & Basis hat eine Ladung von 0\,\% oder weniger.
    & Das Spiel ist beendet, der Spieler, dessen Basis mehr Ladung hat, hat
      gewonnen.
  \\\midrule

  \defid[select-building]{Verteidigungsgebäude auswählen}
    & Spieler
    & Spieler klickt mit der Linken Maustaste in der Liste der
      Verteidigungsgebäude auf ein Gebäude.
    & % Keine Voraussetzung.
    & Spieler befindet sich im Baumodus (siehe Tab.~\ref{tab:build-mode}). Es
      folgt entweder \refid{B:leave} oder \refid{B:choose-position}.
  \\\midrule

  \defid[activate-building]{Gebäude aktivieren}
    & Ge"-bäude
    & Das Gebäude aktiviert sich.
    & Das Gebäude ist entweder \emph{Schockfeld} \textbf{\textls{oder}} an der
      gewählten Position befinden sich keine Einheiten.
    & Das Gebäude ist aktiviert und kann feindliche Einheit angreifen
      (\refid{A:tower-attack}).
  \\\midrule

  \defid[buy-attack]{Einheitenspawner kaufen}
    & Spieler
    & \vspace*{-0.2cm}\begin{enumerate}[nosep,leftmargin=*]
        \item Spieler klickt mit der Linken Maustaste auf eine Einheit in der
          Liste der Angriffseinheiten.
        \item Dem Spieler werden die Kosten der Einheit von seinen Bitcoins
          abgezogen.
        \item Die Zahl neben Button wird um eins inkrementiert.
      \end{enumerate}
    & Der Spieler hat genügend Bitcoin um die Einheit zu kaufen.
    & Beim Beginn aller zukünftigen Wellen spawnt eine Einheit mehr der
      ausgewählten Sorte.
  \\\midrule

  \defid[object-info]{Informationen anzeigen}
    & Spieler
    & Informationen über das Objekt werden eingeblendet.
    & Der Spieler hovert mit dem Mauszeiger über Gebäuden in der Liste der
      Verteidigungsgebäude, Einheiten in der Liste der Angriffseinheiten oder
      einem Upgrade.
    & Informationen werden eingeblendet.

  \\\bottomrule
\end{longtabu}
\endgroup


% ---------------------------------------------------------------------------
% Gebäudeaktionen

\StartId{G}
\begingroup
  \small
  \tabulinesep=1mm
\begin{longtabu}{X[0.6L]X[L]X[L]X[L]}
  \rowfont{\normalsize}
  \caption{Mögliche Aktionen des Spielers auf Gebäuden.\label{tab:building-actions}}\\
  \midrule[\heavyrulewidth]\rowfont{\itshape}
    ID/Name              &
    Ereignis"-fluss      &
    Anfangs"-bedingung   &
    Abschluss"-bedingung \\
  \midrule\endfirsthead

  \rowfont{\normalsize}
  \caption[]{Mögliche Aktionen des Spielers auf Gebäuden (fortges.)}\\
  \midrule[\heavyrulewidth]\rowfont{\itshape}
    ID/Name              &
    Ereignis"-fluss      &
    Anfangs"-bedingung   &
    Abschluss"-bedingung \\
  \midrule\endhead

  \multicolumn{4}{r}{\itshape fortges. auf der nächsten Seite}\\
  \endfoot

  \endlastfoot

  \defid[sell]{Gebäude verkaufen}
    & Der Spieler erhält 80\,\% des Gebäudewertes in Bitcoin zurück und das
      Gebäude verschwindet.
    & Ein eigenes Gebäude ist ausgewählt.
    & Die Kachel, auf der das Gebäude stand, ist nun wieder für gegnerische
      Einheiten passierbar und ein neues Gebäude kann wieder an diese Stelle
      gebaut werden.
  \\\midrule

  \defid[improve]{Gebäude verbessern}
    & Die Kosten der Verbesserung werden von den den Bitcoin des Spielers
      abgezogen und die Werte des Gebäudes werden erhöht.
    & Ein eigenes Gebäude, das verbessert werden kann, ist ausgewählt und der
      Spieler hat genügend Bitcoin.
    & Das ausgewählte Gebäude hat stärkere Werte und die Bitcoin wurden abgezogen.
  \\\midrule

  \defid[choose-strategy]{Strategie wählen}
    & Der Spieler klickt mit der Linken Maustaste auf die gewünschte Strategie.
    & Ein eigenes Gebäude, dass eine Strategie unterstützt, ist ausgewählt.
    & Das Gebäude handelt nun nach der gewählten Strategie.
  \\\bottomrule

\end{longtabu}
\endgroup


% ---------------------------------------------------------------------------
% Baumodus


\StartId{B}
\begingroup
  \small
  \tabulinesep=1mm
\begin{longtabu}{X[0.6L]X[L]X[L]X[L]}
  \rowfont{\normalsize}
  \caption{Optionen und Aktionen des Spielers im Baumodus.\label{tab:build-mode}}\\
  \midrule[\heavyrulewidth]\rowfont{\itshape}
    ID/Name              &
    Ereignis"-fluss      &
    Anfangs"-bedingung   &
    Abschluss"-bedingung \\
  \midrule\endfirsthead

  \rowfont{\normalsize}
  \caption[]{Optionen und Aktionen des Spielers im Baumodus (fortges.)}\\
  \midrule[\heavyrulewidth]\rowfont{\itshape}
    ID/Name              &
    Ereignis"-fluss      &
    Anfangs"-bedingung   &
    Abschluss"-bedingung \\
  \midrule\endhead

  \multicolumn{4}{r}{\itshape fortges. auf der nächsten Seite}\\
  \endfoot

  \endlastfoot

  \defid[leave]{Baumodus verlassen}
    & Spieler klickt mit der Linken Maustaste auf das aktuell gewählte Gebäude
    & Spieler befindet sich im Baumodus.
    & Spieler befindet sich nicht mehr im Baumodus.
  \\\midrule

  \defid[choose-position]{Gebäudeplatzierung wählen}
    & Der Spieler wählt mit der Maus einen gültigen Ort für ein Gebäude.
    & Der Spieler befindet sich im Baumodus.
    & Unter der Annahme, dass das Gebäude an der gewählten Position gebaut
      würde, muss für jede lebende Einheit ein Weg zu jeder Basis frei sein und
      ein Weg zwischen den Basen bestehen. Es folgt entweder \refid{B:leave}
      oder \refid{B:put-defend}.
  \\\midrule

  \defid[put-defend]{Gebäude platzieren}
    & Der Spieler klickt mit der linken Maustaste, das Gebäude wird platziert
      und die Kosten des Gebäudes von den eigenen Bitcoins abgezogen.
    & Der Spieler hat eine gültige Gebäudeplatzierung gewählt
      (\refid{B:choose-position}) und genügend Bitcoin um das Gebäude zu kaufen.
    & Das Bitcoin wurde reduziert und das Gebäude platziert, das auf den
      nächstmöglichen Zeitpunkt seiner Aktivierung wartet
      (\refid{A:activate-building}).
  \\\bottomrule

\end{longtabu}
\endgroup

% ---------------------------------------------------------------------------
% Heldenfähigkeiten

\StartId{H}
\begingroup
  \small
  \tabulinesep=1mm
\begin{longtabu}{X[0.6L]X[0.4L]X[L]X[L]X[L]}
  \rowfont{\normalsize}
  \caption{Beschreibung der Heldenfähigkeiten die durch \refid{A:hero-ability} ausgeführt werden.\label{tab:helden}}\\
  \midrule[\heavyrulewidth]\rowfont{\itshape}
    ID/Name              &
    Held                 &
    Ereignis"-fluss      &
    Anfangs"-bedingung   &
    Abschluss"-bedingung \\
  \midrule\endfirsthead

  \rowfont{\normalsize}
  \caption[]{Beschreibung der Heldenfähigkeiten (fortges.)}\\
  \midrule[\heavyrulewidth]\rowfont{\itshape}
    ID/Name              &
    Held                 &
    Ereignis"-fluss      &
    Anfangs"-bedingung   &
    Abschluss"-bedingung \\
  \midrule\endhead

  \multicolumn{5}{r}{\itshape fortges. auf der nächsten Seite}\\
  \endfoot

  \endlastfoot

  \defid[heal]{Heilen}
    & Settings
    & Heilt die Einheiten, die sich im Radius befinden regelmäßig.
    & Es befinden sich Einheiten im Radius.
    & Einheiten haben mehr LP, oder maximale LP.
  \\\midrule

  \defid[jump]{Gebäude überspringen}
    & Firefox
    & Springt über Einheiten und Gebäude in die gewählte Richtung.
    & Das Zielfeld des Sprungs muss ein gültiges Feld auf der Angriffsbahn
      sein und seit dem letzten Einsetzen ist die Abklingzeit vergangen.
    & Firefox hat sich in die gewählte Richtung bewegt.
  \\\midrule

  \defid[emp]{EMP-Angriff}
    & Blue"-screen
    & Das Verteidigungsgebäude, das am nächsten am Helden ist, wird kurzzeitig
      ausgeschaltet.
    & Die Fähigkeit ist aufgeladen.
    & Das nächste Angriffsgebäude wird ausgeschaltet und der EMP-Angriff bis
      zur Aufladung (\refid{H:reload}) deaktiviert.
  \\\midrule

  \defid[reload]{EMP-Aufladung}
    & Blue"-screen
    & Der EMP-Angriff wird aufgeladen.
    & EMP-Angriff ist deaktiviert und Blue"-screen ist an der eigenen Basis.
    & EMP-Angriff ist wieder aufgeladen.
 \\\bottomrule
\end{longtabu}
\endgroup
