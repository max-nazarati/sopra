\section{Optionen und Aktionen}

% Siehe auch: https://sopra.informatik.uni-freiburg.de/soprawiki/Game_Mechanic

% Dieser Abschnitt beinhaltet die Aktionen, die Spieler oder KI vornehmen
% können, um den Zustand des Spiels zu verändern (zB das Bauen von Einheiten
% oder das Abbauen von Ressourcen). Je klarer diese Aktionen formuliert sind,
% desto leichter fällt einem die Umsetzung der Aktionen bei der Programmierung
% des Spiels.
%
% Wichtig sind auch die Einstellungen, die der Spieler am Spiel vornehmen kann,
% um das Spielverhalten zu verändern (zB.  Schwierigkeitsgrad ändern).
%
% Das Ziel des Spiels sollte schließlich anhand der beschriebenen Aktionen
% erklärt werden.
%
% Die Auflistung der Optionen und Aktionen erfolgt tabellarisch und ist in Form
% und Inhalt an Use Cases (http://de.wikipedia.org/wiki/Use_case) angelehnt.

\missingSection{Optionen und Aktionen}

\StartId{A}
\begingroup
  \small
  \tabulinesep=1mm
\begin{longtabu}{X[0.6L]X[0.4L]X[L]X[L]X[L]}
  \rowfont{\normalsize}
  \caption{Mögliche Optionen und Aktionen\label{tab:optionen-aktionen}}\\
  \midrule[\heavyrulewidth]\rowfont{\itshape}
    ID/Name              &
    Akteure              &
    Ereignis"-fluss      &
    Anfangs"-bedingung   &
    Abschluss"-bedingung \\
  \midrule\endfirsthead

  \rowfont{\normalsize}
  \caption[]{Mögliche Optionen und Aktionen (fortges.)}\\
  \midrule[\heavyrulewidth]\rowfont{\itshape}
    ID/Name              &
    Akteure              &
    Ereignis"-fluss      &
    Anfangs"-bedingung   &
    Abschluss"-bedingung \\
  \midrule\endhead

  \multicolumn{5}{r}{\itshape fortges. auf der nächsten Seite}\\
  \endfoot

  \endlastfoot
  
  \defid[object-selection]{Objekt Auswahl}
  & Spieler
  & \begin{enumerate}[nosep, leftmargin=*]
  \item Spieler clickt auf das Objekt (Turm oder Held).
  \item Objekt ist jetzt ausgewählt.
  \end{enumerate}
  & Objekt ist auf einer der 2 Strecken und gehört dem Spieler.
  & Objekt ist ausgewählt.
  
    \\\midrule
    \defid[cancel-object-selection]{Objekt Auswahl kündigen}
    & Spieler
    & \begin{enumerate}[nosep,leftmargin=*]
    	\item Spieler druckt TASTE, was den Objekt-Auswahl terminiert.
    \end{enumerate}
    & Ein Objekt (Turm oder Held) ist gerade ausgewählt.
    & Objekt nicht mehr ausgewählt.
	\\\midrule
  % Example line taken from the SOPRA wiki.
  \defid[move-figure]{Figur(en) durch Klick bewegen}
    % This can now be referenced using \refid{A:move-figure}
    & Spieler
    & \begin{enumerate}[nosep,leftmargin=*]
    	\item Click innerhalb der Angriffstrecke.
    	\item Figur(en) bewegen sich zur
    	angegebenen Position.
    \end{enumerate}
    & Der Spieler muss eine oder mehr kontrollierbare, auswählbare Spielfiguren
      ausgewählt haben.
    & Die Spielfiguren befinden sich am Zielpunkt, \textbf{oder} die
      Spielfiguren befinden an einem begehbaren Punkt in der Welt, der möglichst
      nah am Zielpunkt liegt.
      
  \\\midrule
  \defid[hero-ability]{Held Fähigkeit Auswahl}
  & Spieler
  & \begin{enumerate}[nosep, leftmargin=*]
  \item click auf Fähigkeit.
  \item Führe Fähigkeit aus
  \end{enumerate}
  & Held ist ausgewählt und kein aktiver cool-down timer für die Fähigkeit.
  & Fähigkeit war ausgeführt.      
  
  \\\midrule
  \defid[select-building]{Gebäude aus Menu auswählen}
  & Spieler
  & \begin{enumerate}[nosep, leftmargin=*]
  \item Gebäude wird durch click ausgewählt
  \end{enumerate}
  & Maus hängt über einen Gebäude-Menueintrag
  & Gebäude ist jetzt ausgewählt und man kann es auf der Verteidigungsstrecke herum ziehen um
  Baustelle auszuwählen.
  
  \\\midrule
  \defid[put-defend]{Gebäude kaufen}
  & Spieler
  &  \begin{enumerate}[nosep, leftmargin=*]
  \item Spieler clickt auf der Verteidigungsstrecke.
  \item Geld für das Gebäude wird abgezogen.
  \item Gebäude wird auf die Ausgewählte Stelle plaziert.
  \end{enumerate}
  & Genug Gold vorhanden,Maus auf der Verteidigungsstrecke und 
  Umgebung nicht belegt.
  & Gebäude wird auf der Position gebaut
  
  \\\midrule
  \defid[buy-attack]{Einheit kaufen.}
  & Spieler
  & \begin{enumerate}[nosep,leftmargin=*]
    \item Spieler clickt.
    \item Gold wird abgezogen.
    \item Einheit wird produziert
    \end{enumerate}
  & \begin{enumerate}[nosep,leftmargin=*]
  \item Mauszeiger zeigt auf eine Einheitskarte im Einheitenmenu.
  \item Genug Gold vorhanden.
  \end{enumerate}
  & Einheit produziert.
  
  \\\midrule
  \defid[update-defend]{Turm updaten}
  & Spieler
  &\begin{enumerate}[nosep, leftmargin=*]
  \item Spieler clickt auf Update.
  \item Turm ist aktualisiert.
  \end{enumerate}
  & Turm war vorher ausgewählt und genug Gold vorhanden.
  & Turm ist up-to-date.
  
    \\\midrule
    \defid[select-Upgrade]{Upgrade auswählen}
    & Spieler
    & \begin{enumerate}[nosep,leftmargin=*]
    	\item click
    	\item Wenn genug XP, Fähigkeit wird eingeschaltet. Sonst, irgendeine Meldung.
    \end{enumerate}
    & Mauszeiger befindet sich über einen Upgrade-Node
    & Upgrade wird angewendet oder Meldung wird angezeigt.  
  
  \\\midrule
  \defid[unit-hit]{Turm wird angegriffen}
  & Turm
  & \begin{enumerate}[nosep, leftmargin=*]
  \item Lebenspunkte werden abgezogen
  \begin{itemize}[nosep, leftmargin=*]
  	\item Wenn Lebenspunkte noch positiv, das war es
  	\item Sonst Turm wird vollständing zerstört
  \end{itemize}
  \end{enumerate}
  & gegnerische Angriffseinheit, greift Turm aktiv an.
  & Turm hat weniger Lebenspunkte, oder Turm ist zerstört.
  
    \\\midrule
    \defid[tower-attack]{Turm Angriff}
    & Turm
    &\begin{enumerate}[nosep, leftmargin=*]
    \item Turm greift Gegner an solange Gegner nah genug.
    \end{enumerate}
    & Gegner in Reichweite vom Turm.
    & Gegner getötet oder entkommen.

    \\\midrule
    \defid[die]{Angriffseinheit stirbt}
    & Einheit
    & \begin{enumerate}[nosep,leftmargin=*]
    \item Einheit stribt (nicht mehr auf dem Spielfeld)
    \end{enumerate}
    & Einheit hat 0 Lebenspunkte.
    & Einheit nicht mehr sichtbar und nicht mehr verfügbar.
    
    \\\midrule
    \defid[damage-base]{Gegner-Basis Angreifen}
    &Angriffs"-einheit.
    & \begin{enumerate}[nosep, leftmargin=*]
    \item Entsprechend viele Lebenspunkte sind von der Gegner-Basis abgezogen
    \item Angriffseinheit wird gelöscht.
    \end{enumerate}
    & Angrffseinheit hat Gegner-Basis erreicht.
    & Gegner-Basis hat weniger Lebenspunkte, Angriffseinheit nicht mehr verfügbar.
    
    \\\midrule
    \defid[base-die]{Basis kaputt}
    & Basis.
    & \begin{enumerate}[nosep, leftmargin=*]
    \item Lebenspunkten sind abgezogen.
    \item \begin{itemize}[nosep, leftmargin=*]
    	\item wenn Lebenspunkte noch positiv, das war es.
    	\item Sonst Basis kaputt, Spiel vorbei.
    \end{itemize}
    \end{enumerate}
    & Gegnerische Angriffseinheit hat Basis erreicht.
    & Basis jetzt mit weniger Lebenspunkte oder kaputt.
  \\

  \bottomrule
\end{longtabu}
\endgroup
