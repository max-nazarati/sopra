\subsection{Angriffseinheiten}\label{sec:attack}

\begin{description}
  \item[Truppen]
    kosten relativ wenig, lassen sich jedoch nicht weiter kontrollieren.
    Sie sind einer Welle zugeordnet und spawnen gemeinsam mit den anderen
    Truppen dieser Welle. Sie verfolgen das Ziel, auf dem kürzesten Weg das
    gegenerischen Lager zu erreichen um dort Schaden zu verursachen.


  \item[Helden] kosten mehr als Truppen, diese Einheiten lassen sich jedoch vom
    Spieler kontrollieren und so strategisch einsetzen und außerhalb der
    Reichweite von Verteidigungsgebäuden positionieren; zusätzlich besitzen sie
    Fähigkeiten, die der Spieler einsetzen kann. Es ist nicht möglich, mehr als
    einen Helden einer Art zum gleichen Zeitpunkt am Leben zu haben. Helden
    sind keiner Welle zugeordnet, sie spawnen direkt beim Kauf.

\end{description}

Tabelle~\ref{tab:attack-unit-props} beschreibt die Eigenschaften die
Angriffseinheiten haben, in Tabelle~\ref{tab:attack-units} sind alle Truppen
mit ihren Eigenschaften aufgelistet und Tabelle~\ref{tab:attack-heroes} enthält
alle Helden.

\begingroup
  \small
  \begin{longtabu}{rlX}
    \rowfont{\normalsize}
    \caption{Eigenschaften von Angriffseinheiten\label{tab:attack-unit-props}}\\

    \midrule[\heavyrulewidth]\rowfont{\itshape}
    & Eigenschaft & Beschreibung \\
    \midrule

    B  & Beschreibung
       & Eine allgemeine Beschreibung dieser Einheit und Vergleich zu anderen
         Einheiten. \\
    F  & Fähigkeit
       & Nur Helden haben eine Fähigkeit, diese kann vom Spieler aktiviert
         werden (\refid{A:hero-ability}). \\
    K  & Kosten
       & Die Menge an Bitcoin die aufgewendet werden muss, um eine dieser
         Einheiten zu kaufen (\refid{A:buy-attack}). \\
    LP & Lebenspunkte
       & Die Zahl der Lebenspunkte einer Einheit: Angriffe von
         Verteidigungstürmen ziehen Lebenspunkte von diesem Wert ab; fällt er
         unter Null, so stirbt diese Einheit (\refid{A:die}). \\
    AS & Angriffsstärke
       & Schaden, den diese Einheit am gegenerischen Lager verursacht, wenn sie
         dieses erreicht (\refid{A:damage-base}). \\
    GS & Geschwindigkeit
       & Distanz, die pro Zeiteinheit zurückgelegt werden kann. \\

    \bottomrule
  \end{longtabu}
  \todo[inline, caption = {Größe und Kollision}]{%
    Größe und Kollision evtl. in die Tabelle aufnehmen, aber was sagt die Größe
    genau aus? Diese ist doch nur interessant, wenn die Einheit kollidiert?}
\endgroup

\begingroup
  \small
  \begin{longtabu}{rXp{0.191\linewidth}}
    \rowfont{\normalsize}
    \caption{Truppen und ihre Werte\label{tab:attack-units}}
    \\\midrule[\heavyrulewidth]\endfirsthead

    % TODO: This seems to introduce too much vertical whitespace between the
    % midrule and the next row.
    \rowfont{\normalsize}
    \caption[]{Truppen und ihre Werte (fortges.)}
    \\\midrule[\heavyrulewidth]\endhead

    \multicolumn{3}{r}{\itshape fortges. auf der nächsten Seite}
    \\\endfoot

    \endlastfoot

    \multicolumn{3}{c}{\bfseries Bug} \\*\midrule
    B  & \emph{(nicht kollidierend)} Eine schnelle Sprintereinheit ohne viele
         Lebenspunkte, die alleine nicht besonders viel Schaden verursacht,
         aber in großer Masse gekauft werden kann, da sie nicht viel kostet.
       & \missingpic \\*
    K  & 1    \\*
    LP & 1    \\*
    AS & 1    \\*
    GS & 10   \\
    \midrule[\heavyrulewidth]

    \multicolumn{3}{c}{\bfseries Virus} \\*\midrule
    B  & \emph{(nicht kollidierend)} Durchschnittliche Einheit, die etwas mehr
         kostet als ein \emph{Bug,} etwas langsamer ist, aber mehr LP hat und
         mehr Schaden verursacht.
       & \missingpic \\*
    K  & 2      \\*
    LP & 2      \\*
    AS & 2      \\*
    GS & 5      \\
    \midrule[\heavyrulewidth]

    \multicolumn{3}{c}{\bfseries Trojaner} \\*\nopagebreak\midrule\nopagebreak
    B  & \emph{(kollidierend)} Stirbt diese Einheit, werden an der Stelle ihres
         Todes \emph{Bugs} und \emph{Viren} gespawnt
         (\refid{A:spawn-children}). Ein Trojaner ist zwar relativ langsam und
         kostet mehr als \emph{Viren,} hat dafür aber mehr LP und mehr AS.
       & \missingpic \\*
    K  & 4 \\*
    LP & 4 \\*
    AS & 4 \\*
    GS & 3 \\
    \midrule[\heavyrulewidth]
    \pagebreak

    \multicolumn{3}{c}{\bfseries Nokia} \\*\midrule
    B  & \emph{(kollidierend)} Diese Einheit ist bei gleichen Kosten zwar
         langsamer als ein \emph{Trojaner,} dafür aber hat sie mehr LP und AS.
       & \missingpic \\*
    K  & 4 \\*
    LP & 6 \\*
    AS & 6 \\*
    GS & 2 \\
    \midrule[\heavyrulewidth]

    \multicolumn{3}{c}{\bfseries Thunderbird} \\*\midrule
    B  & \emph{(kollidierend)} Diese Einheit fliegt, daher muss sie nicht den
         Weg um Mauern und Türme herumfinden, sondern kann einfach auf
         Luftlinie darüber hinwegfliegen.

         Von den Kosten ist diese Einheit mit \emph{Trojaner} vergleichbar, sie
         ist zwar etwas schneller, hat aber nicht viele LP und weniger AS.
       & \missingpic \\*
    K  & 4 \\*
    LP & 4 \\*
    AS & 3 \\*
    GS & 4 \\

    \bottomrule
  \end{longtabu}
  \missingpics{Bilder für Truppen}
\endgroup

\begingroup
  \small
  \begin{longtabu}{rXp{0.191\linewidth}}
    \rowfont{\normalsize}
    \caption{Helden und ihre Werte\label{tab:attack-heroes}}
    \\\midrule[\heavyrulewidth]\endfirsthead

    % TODO: This seems to introduce too much vertical whitespace between the
    % midrule and the next row.
    \rowfont{\normalsize}
    \caption[]{Helden und ihre Werte (fortges.)}
    \\\midrule[\heavyrulewidth]\endhead

    \multicolumn{3}{r}{\itshape fortges. auf der nächsten Seite}
    \\\endfoot

    \endlastfoot

    \multicolumn{3}{c}{\bfseries Settings} \\*\midrule
    B  & \emph{(kollidierend)} Diese Einheit heilt Truppen um sich herum, hat
         jedoch selbst eher wenig LP; diese ist die langsamste der
         Heldeneinheiten, sie verursacht am gegenerischen Lager keinen Schaden.
       & \missingpic \\*
    F  & \emph{(passiv)} heilt verbündete Truppen in Radius 4
         jede Sekunde um 3 LP (\refid{H:heal}).\\*
    K  & 10   \\*
    LP & 4    \\*
    AS & 0    \\*
    GS & 4    \\
    \midrule[\heavyrulewidth]
    \pagebreak

    \multicolumn{3}{c}{\bfseries Firefox} \\*\midrule
    B  & \emph{(kollidierend)} Dieser Held ist eine starke Angriffseinheiten,
         die mit ihrer Fähigkeit leichter zwischen den Verteidigungsgebäuden
         hindurchkommt. Der \emph{Firefox} ist relativ schnell, hat
         durchschnittliche LP und relativ viel~AS.
       & \missingpic \\*
    F  & \emph{(aktiv)} kann 2 Felder überspringen, auch wenn
         Verteidigungsgebäude im Weg stehen (\refid{H:jump}). Die Abklingzeit
         für diese Fähigkeit beträgt fünf Sekunden.\\*
    K  & 10     \\*
    LP & 6      \\*
    AS & 8      \\*
    GS & 8      \\
    \midrule[\heavyrulewidth]

    \multicolumn{3}{c}{\bfseries Bluescreen} \\*\nopagebreak\midrule\nopagebreak
    B  & \emph{(kollidierend)} Diese Einheit unterstützt verbündete Einheit,
         indem sie gegenerische Verteidigungsgebäude für einen Moment
         deaktivieren kann; dafür verursacht sie am gegenerischen Lager selbst
         keinen Schaden, hat wenige LP ist aber schnell.
       & \missingpic \\*
    F  & \emph{(aktiv)} kann eine Schockwelle zünden, um gegenerische
         Verteidigungsgebäude in der Nähe für zwei Sekunden zu deaktivieren
         (\refid{H:emp}).

         Um diese Fähigkeit erneut einzusetzen, muss diese Einheit zur Basis
         zurückkehren um sich aufzuladen (\refid{H:reload}).\\*
    K  & 10     \\*	
    LP & 4      \\*
    AS & 0      \\*
    GS & 10     \\

    \bottomrule
  \end{longtabu}
  \missingpics{Bilder für Helden}
\endgroup
