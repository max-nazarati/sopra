\StartId{H}
\begingroup
  \small
  \tabulinesep=1mm
\begin{longtabu}{X[0.6L]X[0.4L]X[L]X[L]X[L]}
  \rowfont{\normalsize}
  \caption{Beschreibung der Heldenfähigkeiten die durch \refid{A:hero-ability} ausgeführt werden\label{tab:helden}}\\
  \midrule[\heavyrulewidth]\rowfont{\itshape}
    ID/Name              &
    Held                 &
    Ereignis"-fluss      &
    Anfangs"-bedingung   &
    Abschluss"-bedingung \\
  \midrule\endfirsthead

  \rowfont{\normalsize}
  \caption[]{Helden (fortges.)}\\
  \midrule[\heavyrulewidth]\rowfont{\itshape}
    ID/Name              &
    Held                 &
    Ereignis"-fluss      &
    Anfangs"-bedingung   &
    Abschluss"-bedingung \\
  \midrule\endhead

  \multicolumn{5}{r}{\itshape fortges. auf der nächsten Seite}\\
  \endfoot

  \endlastfoot

  \defid[heal]{Heilen}
    & Settings
    & Heilt die Einheiten, die sich im Radius befinden regelmäßig.
    & Es befinden sich Einheiten im Radius.
    & Einheiten haben mehr LP, oder maximale LP.
  \\\midrule

  \defid[jump]{Gebäude überspringen}
    & Firefox
    & Springt über Einheiten und Gebäude in die gewählte Richtung.
    & Das Zielfeld des Sprungs muss ein gültiges Feld auf der Angriffsbahn
      sein.
    & Firefox hat sich in die gewählte Richtung bewegt.
  \\\midrule

  \defid[emp]{EMP-Angriff}
    & Blue"-screen
    & Das Verteidigungsgebäude, das am nächsten am Helden ist, wird kurzzeitig
      ausgeschaltet.
    & Die Fähigkeit ist aufgeladen.
    & Das nächste Angriffsgebäude wird ausgeschaltet und der EMP-Angriff bis
      zur Aufladung (\refid{H:reload}) deaktiviert.
  \\\midrule

  \defid[reload]{EMP-Aufladung}
    & Blue"-screen
    & Der EMP-Angriff wir aufgeladen.
    & EMP-Angriff ist deaktiviert und Blue"-screen ist an der eigenen Basis.
    & EMP-Angriff ist wieder aufgeladen.
 \\\bottomrule
\end{longtabu}
\endgroup
