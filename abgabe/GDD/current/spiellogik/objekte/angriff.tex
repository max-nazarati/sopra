\subsection[Angriffseinheiten]{Angriffseinheiten \hyperref[sec:attack-concept]{\footnotesize $\rightarrow$ Konzept}}
\label{sec:attack}

Tabelle~\ref{tab:attack-unit-props} beschreibt die Eigenschaften die
Angriffseinheiten haben, in Tabelle~\ref{tab:attack-units} sind alle Truppen
mit ihren Eigenschaften aufgelistet und Tabelle~\ref{tab:attack-heroes} enthält
alle Helden.

\begingroup
  \small
  \begin{longtabu}{rlX}
    \rowfont{\normalsize}
    \caption{Eigenschaften von Angriffseinheiten\label{tab:attack-unit-props}}\\

    \midrule[\heavyrulewidth]\rowfont{\itshape}
    & Eigenschaft & Beschreibung \\
    \midrule

    K  & Kosten
       & Die Menge an Bitcoin die aufgewendet werden muss, um eine dieser
         Einheiten zu kaufen (\refid{A:buy-attack}). \\
    LP & Lebenspunkte
       & Die Zahl der Lebenspunkte einer Einheit: Angriffe von
         Verteidigungstürmen ziehen Lebenspunkte von diesem Wert ab; fällt er
         unter Null, so stirbt diese Einheit (\refid{A:die}). \\
    AS & Angriffsstärke
       & Schaden, den diese Einheit am gegenerischen Lager verursacht, wenn sie
         dieses erreicht (\refid{A:damage-base}). \\
    GS & Geschwindigkeit
       & Distanz, die pro Zeiteinheit zurückgelegt werden kann. \\

    \bottomrule
  \end{longtabu}
  \todo[inline, caption = {Größe und Kollision}]{%
    Größe und Kollision evtl. in die Tabelle aufnehmen, aber was sagt die Größe
    genau aus? Diese ist doch nur interessant, wenn die Einheit kollidiert?}
\endgroup

\begingroup
  \small
  \begin{longtabu}{XXXX}
    \rowfont{\normalsize}
    \caption{Truppen und ihre Werte\label{tab:attack-units}}
    \\\midrule[\heavyrulewidth]\endfirsthead

    \rowfont{\normalsize}
    \caption[]{Truppen und ihre Werte (fortges.)}
    \\\midrule[\heavyrulewidth]\endhead

    \multicolumn{4}{r}{\itshape fortges. auf der nächsten Seite}
    \\\endfoot

    \endlastfoot

    \multicolumn{4}{c}{\bfseries Bug} \\*
    K: 2 & LP: 4 & AS: 1 & GS: 7 \\\midrule
    \multicolumn{4}{p{\bodywidth}}{%
      \begin{minipage}{0.85\linewidth}
        Eine schnelle Sprintereinheit ohne viele Lebenspunkte, die alleine
        nicht besonders viel Schaden verursacht, aber in großer Masse gekauft
        werden kann, da sie nicht viel kostet.
      \end{minipage}\hfill\parbox{0.15\linewidth}{%
        \hfill\includegraphics[scale=0.9]{angriff/bug.png}\hfill\null}}
    \\\bottomrule

    \multicolumn{4}{c}{\bfseries Virus} \\*
    K: 3 & LP: 10 & AS: 2 & GS: 5 \\\midrule
    \multicolumn{4}{p{\bodywidth}}{%
      \begin{minipage}{0.85\linewidth}
         Durchschnittliche Einheit, die etwas mehr kostet als ein \emph{Bug,}
         etwas langsamer ist, aber mehr LP hat und mehr Schaden verursacht.
      \end{minipage}\hfill\parbox{0.15\linewidth}{%
        \hfill\includegraphics[scale=0.7]{angriff/virus.png}\hfill\null}}
    \\\bottomrule\pagebreak

    \multicolumn{4}{c}{\bfseries Trojaner} \\*
    K: 20 & LP: 30 & AS: 6 & GS: 3 \\\midrule
    \multicolumn{4}{p{\bodywidth}}{%
      \begin{minipage}{0.85\linewidth}
         Stirbt diese Einheit, werden an der Stelle ihres Todes fünf \emph{Bugs}
         gespawnt (\refid{A:spawn-children}). Ein Trojaner ist zwar relativ
         langsam und kostet mehr als \emph{Viren,} hat dafür aber mehr LP und
         mehr AS.
      \end{minipage}\hfill\parbox{0.15\linewidth}{%
        \hfill\includegraphics[scale=0.6]{angriff/trojan.png}\hfill\null}}
    \\\bottomrule

    \multicolumn{4}{c}{\bfseries Nokia} \\*
    K: 30 & LP: 100 & AS: 15 & GS: 2 \\\midrule
    \multicolumn{4}{p{\bodywidth}}{%
      \begin{minipage}{0.85\linewidth}
         Diese Einheit ist bei gleichen Kosten zwar langsamer als ein
         \emph{Trojaner,} dafür aber hat sie mehr LP und AS.
      \end{minipage}\hfill\parbox{0.15\linewidth}{%
        \hfill\includegraphics[scale=0.6]{angriff/nokia.png}\hfill\null}}
    \\\bottomrule

    \multicolumn{4}{c}{\bfseries Thunderbird} \\*
    K: 15 & LP: 15 & AS: 3 & GS: 3 \\\midrule
    \multicolumn{4}{p{\bodywidth}}{%
      \begin{minipage}{0.85\linewidth}
         Diese Einheit fliegt, daher muss sie nicht den
         Weg um Mauern und Türme herumfinden, sondern kann einfach auf
         Luftlinie darüber hinwegfliegen.

         Von der Geschwindigkeit ist diese Einheit mit \emph{Trojaner}
         vergleichbar, sie ist zwar etwas günstiger, und hat nur halb so viele LP.
      \end{minipage}\hfill\parbox{0.15\linewidth}{%
        \hfill\includegraphics[scale=0.6]{angriff/thunderbird.png}\hfill\null}}
    \\\bottomrule
  \end{longtabu}
\endgroup

\enlargethispage{1cm}
\begingroup
  \small
  \begin{longtabu}{XXXX}
    \rowfont{\normalsize}
    \caption{Helden und ihre Werte\label{tab:attack-heroes}}
    \\\midrule[\heavyrulewidth]\endfirsthead

    \rowfont{\normalsize}
    \caption[]{Helden und ihre Werte (fortges.)}
    \\\midrule[\heavyrulewidth]\endhead

    \multicolumn{4}{r}{\itshape fortges. auf der nächsten Seite}
    \\\endfoot

    \endlastfoot

    \multicolumn{4}{c}{\bfseries Firefox} \\*
    K: 50 & LP: 30 & AS: 10 & GS: 6 \\\midrule
    \multicolumn{4}{p{\bodywidth}}{%
      \begin{minipage}{0.85\linewidth}
        Dieser Held ist eine starke Angriffseinheiten, die mit ihrer Fähigkeit
        leichter zwischen den Verteidigungsgebäuden hindurchkommt. Der
        \emph{Firefox} ist relativ schnell, hat durchschnittliche LP und
        relativ viel~AS.

        \emph{Fähigkeit:} kann 2 Felder überspringen, auch wenn
          Verteidigungsgebäude im Weg stehen (\refid{H:jump}). Die Abklingzeit
          für diese Fähigkeit beträgt fünf Sekunden.
      \end{minipage}\hfill\parbox{0.15\linewidth}{%
        \hfill\includegraphics[scale=0.7]{angriff/firefox.png}\hfill\null}}
    \\\bottomrule

    \multicolumn{4}{c}{\bfseries Bluescreen} \\*
    K: 50 & LP: 15 & AS: 0 & GS: 9 \\\midrule
    \multicolumn{4}{p{\bodywidth}}{%
      \begin{minipage}{0.85\linewidth}
        Diese Einheit unterstützt verbündete Einheit, indem sie gegenerische
        Verteidigungsgebäude für einen Moment deaktivieren kann; dafür
        verursacht sie am gegenerischen Lager selbst keinen Schaden, hat wenige
        LP ist aber schnell.

        \emph{Fähigkeit:} kann eine Schockwelle zünden, um gegenerische
          Verteidigungsgebäude in der Nähe für zwei Sekunden zu deaktivieren
          (\refid{H:emp}).

         Um diese Fähigkeit erneut einzusetzen, muss diese Einheit zur Basis
         zurückkehren um sich aufzuladen (\refid{H:reload}).
      \end{minipage}\hfill\parbox{0.15\linewidth}{%
        \hfill\includegraphics[scale=0.6]{angriff/bluescreen.png}\hfill\null}}
    \\\bottomrule

    \multicolumn{4}{c}{\bfseries Settings} \\*
    K: 50 & LP: 25 & AS: 0 & GS: 4 \\\midrule
    \multicolumn{4}{p{\bodywidth}}{%
      \begin{minipage}{0.85\linewidth}
        Diese Einheit heilt andere Truppen, hat dafür jedoch eher wenig LP;
        dies ist die langsamste Heldeneinheit, sie verursacht am gegenerischen
        Lager keinen Schaden.

        \emph{Fähigkeit:} Heilt alle zwei Sekunden, Truppen im Radius von 1
        Kachel um 2 LP. Dies ist eine passive Fähigkeit.
      \end{minipage}\hfill\parbox{0.15\linewidth}{%
        \hfill\includegraphics[scale=0.6]{angriff/settings.png}\hfill\null}}
    \\\bottomrule

  \end{longtabu}
\endgroup
