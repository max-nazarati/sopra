\subsection{Wellen}

\subsubsection{Terminologie}

Eine Welle besteht aus zwei \emph{Wellenteilen.} Jedes Wellenteil enthält
Truppen eines Spielers.  Sind alle Einheiten eines Wellenteils entweder
gestorben oder erfolgreich in die gegnerische Basis eingetrungen, so gilt diese Teil als „vom Gegner \emph{besiegt}“.
\\
Zwei Wellenteile, nämlich das des Spielers und das des Gegenspielers, bilden
zusammen eine \emph{Welle.} Eine gestartete Wellen, bei der mindestens ein Teil
noch nicht besiegt wurde, heißt \emph{aktiv}.
Solange von einer Welle noch kein Teil besiegt wurde, nennen wir sie die \emph{aktuelle} Welle.


\subsubsection{Wellenstart}

Die erste Welle startet dreißig Sekunden nach Beginn des Spiels. So hat man also eine halbe Minute um sich Vorzubereiten.
Für alle weiteren Wellen gilt folgende Regelung: Die nächste Welle startet in dem
Moment, in dem von der aktuellen Welle, ein Teil besiegt wird.


\subsubsection{Zusammensetzung}

In dem Zeitraum, der mit dem Start einer Welle beginnt und mit der Start der
nächsten Welle endet, können vom Spieler Tuppen gekauft werden, die zu der nun
folgenden Welle gehören, also mit dem Start dieser Welle sich auf den Weg zur
gegnerischen Basis machen.

Ist noch keine Welle gestartet worden, läuft dieser Zeitraum über die ersten
dreißig Sekunden.

Helden sind keiner Welle zugeordnet. Beim Kauf einer solchen Einheit wird sie
direkt an der eigenen Basis gespawnt.
