\section{Spielobjekte}

% Hier sind alle Spielobjekte aufgelistet, die es im Spiel gibt. Dies können
% zum Beispiel Einheiten, Gebäude, Hindernisse usw. sein. Hilfreich zum
% Verständnis und zur Identifizierung der einzelnen Objekte können Bilder sein.
%
% Insbesondere erklärt dieser Abschnitt alle Eigenschaften und Fähigkeiten der
% unterschiedlichen Objekte. Sind sie zum Beispiel einfach zu zerstören, kosten
% sie viele Ressourcen, usw.
%
% Spielobjekte können hier auch schon mit konkreten Werten versehen werden;
% wenn noch keine konkreten Werte vorliegen (z.B. wie viele Lebenspunkte eine
% bestimmte Einheit hat), sollten zumindest die Verhältnisse der Werte von
% unterschiedlichen Spileobjekten aufgelistet werden, also zum Beispiel: Welche
% Einheit ist stärker als eine andere? Dass sich konkrete Werte noch verändern
% können, ist selbstverständlich, und eine Frage des Balancings gegen Ende des
% Projekts.

\missingSection{Spielobjekte}

% \begin{table}[ht!]
%   \caption{Eigenschaftswerte von \dots}
%   \small
%   \begin{longtabu}{cX[-1]X}
%     \toprule\rowfont{\itshape}
%     & Eigenschaft & Beschreibung \\
%     \midrule
%
%     % Example
%     LP & Lebenspunkte & \dots \\
%
%     \bottomrule
%   \end{longtabu}
% \end{table}
%
%
% \begin{table}[ht!]
%   \caption{Werte von Einheiten}
%   \small
%   \begin{longtabu}{cXl}
%     \toprule
%
%     \multicolumn{3}{c}{\bfseries Frosch} \\\midrule
%     Beschreibung & \dots & \itshape Bild hier? \\
%     LP & 1 \\
%     \midrule[\heavyrulewidth]
%
%     \multicolumn{3}{c}{\bfseries Dummy} \\\midrule
%     Beschreibung & \dots & \itshape Bild hier? \\
%     LP & 2 \\
%     \bottomrule
%   \end{longtabu}
% \end{table}


\subsection{Angriffseinheiten}

\begin{description}
  \item[Truppen]\todo[noline]{Benennung überdenken}
    kosten relativ wenig, lassen sich jedoch nicht weiter kontrollieren. Diese
    Einheiten verfolgen das Ziel, möglichst schnell zum gegenerischen Lager zu
    gelangen um dort Schaden zu verursachen.

    In Tabelle~\ref{tab:attack-unit-props} werden die Eigenschaften jeder
    Truppeneinheit beschrieben, Tabelle~\ref{tab:attack-units} enthält alle
    unterschiedlichen Einheiten und weist den Eigenschaften entsprechende Werte
    zu.

  \item[Helden] kosten mehr als Truppen, diese Einheiten lassen sich jedoch vom
    Spieler kontrollieren. Sie besitzen zusätzlich noch Fähigkeiten die der
    Spieler einsetzen kann.

    In Tabelle~\ref{tab:attack-hero-props} werden die Eigenschaften jeder
    Heldeneinheit beschrieben, Tabelle~\ref{tab:attack-heroes} enthält die
    unterschiedlichen Helden und weist den Eigenschaften entsprechende Werte
    zu.
\end{description}

\begin{table}[htbp]
  \caption{Eigenschaften von Truppen}
  \label{tab:attack-unit-props}
  \small
  \begin{longtabu}{rlX}
    \toprule\rowfont{\itshape}
    & Eigenschaft & Beschreibung \\
    \midrule

    LP & Lebenspunkte
       & Die Zahl der Lebenspunkte einer Einheit: Angriffe von
         Verteidigungstürmen ziehen Lebenspunkte von diesem Wert ab; fällt er
         unter Null, so stirbt diese Einheit (\refid{A:die}). \\
    AS & Angriffsstärke
       & Schaden, den diese Einheit am gegenerischen Lager verursacht, wenn sie
         dieses erreicht (\refid{A:damage-base}). \\
    GS & Geschwindigkeit & Distanz, die pro Zeiteinheit zurückgelegt werden
         kann. \\
    K  & Kosten
       & Die Menge an Gold die aufgewendet werden muss, um eine dieser
         Einheiten zu kaufen (\refid{A:buy-attack}). \\

    \bottomrule
  \end{longtabu}
  \todo[inline]{Größe und Kollision evtl. in die Tabelle aufnehmen}
\end{table}

\begingroup
  \small
  \begin{longtabu}{rXp{0.191\linewidth}}
    \rowfont{\normalsize}
    \caption{Truppen und ihre Werte\label{tab:attack-units}}
    \\\midrule[\heavyrulewidth]\endfirsthead

    % TODO: This seems to introduce too much vertical whitespace between the
    % midrule and the next row.
    \rowfont{\normalsize}
    \caption[]{Truppen und ihre Werte (fortges.)}
    \\\midrule[\heavyrulewidth]\endhead

    % TODO: At the moment this leads to strange page breaks, revisit when there
    % is more content!
    %
    % \multicolumn{3}{r}{\itshape fortges. auf der nächsten Seite}
    % \\\endfoot
    %
    % \endlastfoot

    \multicolumn{3}{c}{\bfseries Bug} \\*\midrule
      & Eine schnelle Sprintereinheit ohne viele Lebenspunkte, die alleine
        nicht besonders viel Schaden verursacht, aber in großer Masse gekauft
        werden kann, da sie nicht viel kostet.
      & \itshape Bild hier? \\*
    LP & 1    \\*
    AS & 1    \\*
    GS & 10   \\*
    K  & 1    \\
    \midrule[\heavyrulewidth]

    \multicolumn{3}{c}{\bfseries Virus} \\*\midrule
      & Durchschnittliche Einheit, die etwas mehr kostet als ein \emph{Bug,}
        etwas langsamer ist, aber mehr LP hat und mehr Schaden verursacht.
      & \itshape Bild hier? \\*
    LP & 2      \\*
    AS & 2      \\*
    GS & 5      \\*
    K  & 2      \\
    \midrule[\heavyrulewidth]

    \multicolumn{3}{c}{\bfseries Trojaner} \\*\nopagebreak\midrule\nopagebreak
      & Stirbt diese Einheit, werden an der Stelle ihres Todes \emph{Bugs} und
        \emph{Viren} gespawnt. Ein Trojaner ist zwar relativ langsam und kostet
        mehr als \emph{Viren,} hat dafür aber mehr LP und mehr AS.
      & \itshape Bild hier? \\*
    LP & 4 \\*
    AS & 4 \\*
    GS & 3 \\*
    K  & 4 \\
    \midrule[\heavyrulewidth]

    \multicolumn{3}{c}{\bfseries Nokia} \\*\midrule
      & Diese Einheit ist bei gleichen Kosten zwar langsamer als ein
        \emph{Trojaner,} dafür aber hat sie mehr LP und AS.
      & \itshape Bild hier? \\*
    LP & 6 \\*
    AS & 6 \\*
    GS & 2 \\*
    K  & 4 \\
    \midrule[\heavyrulewidth]

    \multicolumn{3}{c}{\bfseries Thunderbird} \\*\midrule
      & Diese Einheit fliegt, daher muss sie nicht den Weg um Mauern und Türme
        herumfinden, sondern kann einfach auf Luftlinie darüber hinwegfliegen.

        Von den Kosten ist diese Einheit mit \emph{Trojaner} vergleichbar, sie
        ist ebenso relativ langsam, hat aber nicht viele LP und weniger AS.
      & \itshape Bild hier? \\*
    LP & 4 \\*
    AS & 3 \\*
    GS & 3 \\*
    K  & 4 \\

    \bottomrule
  \end{longtabu}
  \todo[inline]{Bilder für Truppen}
\endgroup


\subsection{Verteidigungsgebäude}


