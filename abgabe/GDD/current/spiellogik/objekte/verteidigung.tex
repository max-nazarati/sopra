\subsection{Verteidigungsgebäude}

In Tabelle~\ref{tab:defend-props} werden die Eigenschaften von
Verteidigungsgebäuden beschrieben, Tabelle~\ref{tab:defend-units} enthält die
Gebäude und weist den Eigenschaften Werte zu. Der Wert W berechnet sich aus der
Menge an Bitcoin, die in dieses Gebäude investiert wurde.

Im Laufe des Spiels kann der Spieler folgende Aktionen auf eigenen Türmen
ausführen:

\begin{description}
  \item[Verkaufen] (\refid{A:tower-sell}) Das Gebäude verschwindet, es können
    neue Gebäude an dieser Stelle gebaut werden und feindliche Einheiten können
    wieder über diese Felder laufen.

    Der Spieler erhält 80\,\% des Gebäudewertes an Bitcoin.

  \item[Verbessern] (\refid{A:tower-improve}) Erhöht die Reichweite des
    Gebäudes (außer bei \emph{Schockfeld}) um 50\,\% des aktuellen Wertes und
    reduziert das Angriffsintervall um 20\,\% des aktuellen Wertes.

    Die Verbesserung kostet den Spieler 50\,\% des aktuellen Turmwertes und der
    Turmwert steigt um diese Kosten.

    Bei \emph{Kabel} ist keine Verbesserung möglich, jedes andere Gebäude kann
    maximal zweimal verbessert werden.

  \item[Strategie wählen] (\refid{A:tower-strategy}) Mögliche Strategien sind
    \begin{description}[itemsep=0pt]
      \item[Erste Einheit] \emph{(ist standardmäßig ausgewählt)}\\
        Greift die Einheit an, die den kürzesten Weg hat, um Schaden an der Basis
        zu verusachen.

      \item[Stärkste Einheit] ~\\
        Greift die Einheit an, die die meisten LP hat.

      \item[Schwächste Einheit] ~\\
        Greift die Einheit an, die die wenigsten LP hat.
    \end{description}

    Bei \emph{Kabel} und \emph{Schockfeld} ist ein Wählen der Strategie nicht
    möglich.

\end{description}


\begingroup
  \small
  \begin{longtabu}{rlX}
    \rowfont{\normalsize}
    \caption{Eigenschaften von Verteidigungsgebäuden\label{tab:defend-props}}\\

    \midrule[\heavyrulewidth]\rowfont{\itshape}
    & Eigenschaft & Beschreibung \\
    \midrule

    B  & Beschreibung
       & Eine allgemeine Beschreibung dieser Einheit und Vergleich zu anderen
         Einheiten. \\
    K  & Kosten
       & Die Menge an Bitcoin die aufgewendet werden muss, um eines dieser
         Gebäude zu platzieren~(\refid{A:put-defend}). \\
    VS & Verteidigungsstärke
       & Schaden, den dieses Gebäude an getroffenen Gegner
         verursacht~(\refid{A:unit-hit}). \\
    AI & Angriffsintervall
       & Zeit die vergehen muss, bevor dieses Gebäude erneut Gegner angreifen
         kann~(\refid{A:tower-attack}). \\
    RW & Reichweite
       & Radius um den Turm, in dem Einheiten angegriffen werden können, und in
         dem die Effekte der Türme auf die Einheiten wirken. \\

    \bottomrule
  \end{longtabu}
\endgroup

\begingroup
  \small
  \begin{longtabu}{rXp{0.191\linewidth}}
    \rowfont{\normalsize}
    \caption{Verteidigungsgebäude und ihre Werte\label{tab:defend-units}}
    \\\midrule[\heavyrulewidth]\endfirsthead

    \rowfont{\normalsize}
    \caption[]{Verteidigungsgebäude und ihre Werte (fortges.)}
    \\\midrule[\heavyrulewidth]\endhead

    \multicolumn{3}{r}{\itshape fortges. auf der nächsten Seite}
    \\\endfoot

    \endlastfoot

    \multicolumn{3}{c}{\bfseries Kabel} \\*\midrule
    B  & Dieses Gebäude kostet wenig, steht gegnerischen Einheiten im Weg und
         verursacht keinen Schaden.
       & \missingpic \\*
    K  & 2 \\*
    VS & --- \\*
    AI & --- \\*
    RW & --- \\
    \midrule[\heavyrulewidth]

    \multicolumn{3}{c}{\bfseries Mauszeigerschütze} \\*\midrule
    B  & Durchschnittlicher Verteidigungsturm, der Mauszeiger auf ein
         Einzelziel verschießt.
       & \missingpic \\*
    K  & 3 \\*
    VS & 1 \\*
    AI & 1 \\*
    RW & 4 \\
    \midrule[\heavyrulewidth]

    \multicolumn{3}{c}{\bfseries CD-Werfer} \\*\midrule
    B  & Dieser Turm kostet mehr und schießt langsamer als ein
         \emph{Mauszeigerschütze,} dafür verursacht das Projektil (die CD)
         jedoch auf ihrem Weg an jedem berührten Gegner den Schaden der Höhe
         VS.
       & \missingpic \\*
    K  & 5 \\*
    VS & 4 \\*
    AI & 3 \\*
    RW & 3 \\
    \midrule[\heavyrulewidth]

    \multicolumn{3}{c}{\bfseries Antivirusprogramm} \\*\midrule
    B  & Von den Kosten ist dieser Turm vergleichbar zum \emph{CD-Werfer,}
         allerdings schießt das \emph{Antivirusprogramm} noch langsamer,
         verursacht dafür aber an einem Einzelziel erheblichen Schaden.
       & \missingpic \\*
    K  & 5 \\*
    VS & 7 \\*
    AI & 5 \\*
    RW & 6 \\
    \midrule[\heavyrulewidth]

    \multicolumn{3}{c}{\bfseries Lüftung} \\*\midrule
    B  & Dieser Turm verlangsamt alle Einheiten im Einflussbereich.
       & \missingpic \\*
    K  & 5 \\*
    VS & 7 \\*
    AI & 5 \\*
    RW & 4 \\
    \midrule[\heavyrulewidth]
    \pagebreak

    \multicolumn{3}{c}{\bfseries Wifi-Router} \\*\midrule
    B  & Dieser Turm schießt nahezu dauerhaft kreisförmige Wellen, die wenig
         Schaden verursachen und Gegner penetrieren.
       & \missingpic \\*
    K  & 5 \\*
    VS & 2 \\*
    AI & 1 \\*
    RW & 5 \\
    \midrule[\heavyrulewidth]

    \multicolumn{3}{c}{\bfseries Schockfeld} \\*\midrule
    B  & Dieses „Gebäude“ blockiert die Gegner nicht, sie laufen darüber
         hinweg. In regelmäßigen Abständen erhalten alle Gegner schaden, die
         auf einem \emph{Schockfeld} sind.
       & \missingpic \\*
    K  & 4 \\*
    VS & 2 \\*
    AI & 3 \\*
    RW & 0 \\

    \bottomrule
  \end{longtabu}
  \missingpics{Bilder für Verteidigungsgebäuden}
\endgroup
