\section{Tastaturbelegung}

Das Spiel lässt sich vollständig intuitiv mit der Maus steuern.
Um Aktionen schneller durchführen zu können ist dennoch die Tastatur hilfreich.
Manche Tasten sind mit mehreren Optionen belegt. Es wird die zu erst genannte Option (weiter oben in der Tabelle) gewählt, deren Vorbedingungen erfüllt sind.\\
Die Tastaturbelegung lässt sich im Optionsmenü
(Abschnitt~\ref{sec:menu-options}) anpassen. Standardmäßig sind die Tasten wie
in Tabelle~\ref{tab:tastaturbelegung} belegt.

\begingroup
\small
\tabulinesep=1.2mm
\begin{longtabu}{X[0.3]X[0.6]X[L]}
	\rowfont{\normalsize}
	\caption{Standardtastaturbelegung in \emph{Kernel Panic!}.\label{tab:tastaturbelegung}}\\
	\midrule[\heavyrulewidth]\rowfont{\itshape}
	Taste & Vorbedingung & Aktion \\
	\midrule\endfirsthead
	
	\rowfont{\normalsize}
	\caption[]{Tastaturbelegung (fortges.)}\\
	\midrule[\heavyrulewidth]\rowfont{\itshape}
	Taste & Vorbedingung & Aktion \\
	\midrule\endhead
	
	\bottomrule
	\multicolumn{2}{r}{\emph{fortges. auf der nächsten Seite}} \\
	\endfoot
	
	\endlastfoot
	
	% Example
	\emph{Escape}-Taste
	& Im Spiel, Auswahl an Einheiten / Türmen vorhanden
	& Auswahl aufheben
	\\
	%
	\emph{Escape}-Taste
	& Im Spiel, Auswahl an Einheiten / Türmen vorhanden
	& Auswahl aufheben
	\\
	%
	\emph{Escape}-Taste
	& Im Spiel, keine Auswahl
	& Pause-Menü öffnen
	\\
	%
	\emph{Escape}-Taste
	& Im Pause-Menü
	& Pause-Menü schließen / weiter Spielen
	\\
	% 
	W
	& Im Spiel
	& Kamera nach oben bewegen
	\\
	%
	A
	& Im Spiel
	& Kamera nach links bewegen
	\\
	%
	S
	& Im Spiel
	& Kamera nach unten bewegen
	\\
	%
	D
	& Im Spiel
	& Kamera nach rechts bewegen
	\\
	%
	Q
	& Held ausgewählt, Fähigkeit wird nicht indiziert
	& Fähigkeit wird indiziert
	\\
	%
	Q
	& Held ausgewählt, Fähigkeit wird indiziert
	& Fähigkeit wird bestätigt TODO Link auf Tabelle
	\\
	%
	E
	& Held ausgewählt, Fähigkeit wird indiziert
	& Fähigkeitsindizierung abbrechen. TODO Link auf Tabelle
	\\
	\bottomrule
\end{longtabu}
