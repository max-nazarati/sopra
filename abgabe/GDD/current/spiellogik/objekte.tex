\section{Spielobjekte}

% Hier sind alle Spielobjekte aufgelistet, die es im Spiel gibt. Dies können
% zum Beispiel Einheiten, Gebäude, Hindernisse usw. sein. Hilfreich zum
% Verständnis und zur Identifizierung der einzelnen Objekte können Bilder sein.
%
% Insbesondere erklärt dieser Abschnitt alle Eigenschaften und Fähigkeiten der
% unterschiedlichen Objekte. Sind sie zum Beispiel einfach zu zerstören, kosten
% sie viele Ressourcen, usw.
%
% Spielobjekte können hier auch schon mit konkreten Werten versehen werden;
% wenn noch keine konkreten Werte vorliegen (z.B. wie viele Lebenspunkte eine
% bestimmte Einheit hat), sollten zumindest die Verhältnisse der Werte von
% unterschiedlichen Spileobjekten aufgelistet werden, also zum Beispiel: Welche
% Einheit ist stärker als eine andere? Dass sich konkrete Werte noch verändern
% können, ist selbstverständlich, und eine Frage des Balancings gegen Ende des
% Projekts.

\missingSection{Spielobjekte}

\begin{table}[ht!]
  \caption{Eigenschaftswerte von \dots}
  \small
  \begin{longtabu}{cX[-1]X}
    \toprule\rowfont{\itshape}
    & Eigenschaft & Beschreibung \\
    \midrule

    % Example
    LP & Lebenspunkte & \dots \\

    \bottomrule
  \end{longtabu}
\end{table}


\begin{table}[ht!]
  \caption{Werte von Einheiten}
  \small
  \begin{longtabu}{cXl}
    \toprule

    \multicolumn{3}{c}{\bfseries Frosch} \\\midrule
    Beschreibung & \dots & \itshape Bild hier? \\
    LP & 1 \\
    \midrule[\heavyrulewidth]

    \multicolumn{3}{c}{\bfseries Dummy} \\\midrule
    Beschreibung & \dots & \itshape Bild hier? \\
    LP & 2 \\
    \bottomrule
  \end{longtabu}
\end{table}
