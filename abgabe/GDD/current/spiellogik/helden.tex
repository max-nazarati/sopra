\section{Helden}

% Siehe auch: https://sopra.informatik.uni-freiburg.de/soprawiki/Game_Mechanic

% Dieser Abschnitt beinhaltet die Aktionen, die Spieler oder KI vornehmen
% können, um den Zustand des Spiels zu verändern (zB das Bauen von Einheiten
% oder das Abbauen von Ressourcen). Je klarer diese Aktionen formuliert sind,
% desto leichter fällt einem die Umsetzung der Aktionen bei der Programmierung
% des Spiels.
%
% Wichtig sind auch die Einstellungen, die der Spieler am Spiel vornehmen kann,
% um das Spielverhalten zu verändern (zB.  Schwierigkeitsgrad ändern).
%
% Das Ziel des Spiels sollte schließlich anhand der beschriebenen Aktionen
% erklärt werden.
%
% Die Auflistung der Optionen und Aktionen erfolgt tabellarisch und ist in Form
% und Inhalt an Use Cases (http://de.wikipedia.org/wiki/Use_case) angelehnt.

\missingSection{Helden in Tabelle}

\StartId{H}
\begingroup
  \small
  \tabulinesep=1mm
\begin{longtabu}{X[0.6L]X[0.4L]X[L]X[L]X[L]}
  \rowfont{\normalsize}
  \caption{Helden\label{tab:helden}}\\
  \midrule[\heavyrulewidth]\rowfont{\itshape}
    ID/Name              &
    Akteure              &
    Ereignis"-fluss      &
    Anfangs"-bedingung   &
    Abschluss"-bedingung \\
  \midrule\endfirsthead

  \rowfont{\normalsize}
  \caption[]{Helden (fortges.)}\\
  \midrule[\heavyrulewidth]\rowfont{\itshape}
    ID/Name              &
    Akteure              &
    Ereignis"-fluss      &
    Anfangs"-bedingung   &
    Abschluss"-bedingung \\
  \midrule\endhead

  \multicolumn{5}{r}{\itshape fortges. auf der nächsten Seite}\\
  \endfoot

  \endlastfoot

  % Example line taken from the SOPRA wiki.
  \defid[settings-hero]{Ubuntu-settings}
    % This can now be referenced using \refid{A:move-figure}
    & Spieler
    & Ubuntu settings heilt alle Einheiten, die sich in einem bestimmten Radius befinden regelmäßig um einen bestimmten Wert.
    & Einheit befindet sich in Heilradius von Ubuntu-settings.
    & Einheit verlässt Heilradius von Ubuntu-settings.
  \\\midrule
  \defid[firefox-hero]{Firefox}
    & Spieler
    & Der Firefox kann gegnerische Verteidigungsgebäude überspringen.
    & Firefox läuft auf gegnerisches Verteidigungsgebäude zu.
    & Firefox erreicht gegnerischen Verteidigungsgebäude und überspringt es.
  \\\midrule
  \defid[bluescreen-hero]{Bluescreen}
    & Spieler
    & Der Bluescreen kann eine Schockwelle zünden, mit der er gegnerische Verteidigungsgebäude vorübergehend deaktiviert.
    & Bluescreen wird von Spieler in die Nähe gegnerischer Türme gesteuert. Spieler löst Schockwelle aus, und gegnerische Türme werden deaktiviert.
    & Bluescreen muss vom Spieler zurück zur Basis gesteuert werden, wo sich seine Spezialfähigkeit wieder auflädt.
    
    \\\midrule
    \defid[reload]{Bluescreen reload}
    & Blue"-screen
    & \vspace*{-0.24cm}\begin{enumerate}[nosep, leftmargin=*]
    \item Bluescreen erreicht eigene Basis.
    \item EMP Fähigkeit wieder verfügbar.
    \end{enumerate}
    & \vspace*{-0.24cm}\begin{itemize}[nosep, leftmargin=*]
    \item EMP war schon verwendet.
    \item Spieler hat Bluescreen ausgewählt und zurück zur Basis geschickt.
    \end{itemize}
    & EMP Fähigkeit kann von Spieler ausgewählt werden.
    
   \\\midrule
   \defid[emp]{Bluescreen EMP Angriff}
   & Blue"-screen
   & \vspace*{-0.24cm}\begin{enumerate}[nosep, leftmargin=*]
   \item Bluescreen macht den EMP Angriff.
   \item Alle Türme die nah genug sind werden kurzzeitig ausgeschaltet.
   \end{enumerate}
   & Spieler hat die EMP Fähigkeit von Bluescreen ausgewählt.
   & Nah liegende Türme werden betroffen, Fähigkeit muss wieder bei der Basis entschlossen werden.
   \\\midrule
  \bottomrule
\end{longtabu}
\endgroup
