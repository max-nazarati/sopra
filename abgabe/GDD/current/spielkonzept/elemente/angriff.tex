\subsection{Angriffseinheiten}
% \subsection[Angriffseinheiten]{Angriffseinheiten {\small (Spielobjekte in~\ref{sec:attack})}}

Die Angriffseinheiten lassen sich in zwei Gruppen einteilen. Auf der einen
Seite stehen die autonomen \emph{Truppen} und auf der anderen die
kontrollierbaren \emph{Helden.} Gemeinsam bilden sie die
\emph{Angriffseinheiten,} oder kurz \emph{Einheiten.} Alle Einheiten spawnen
auf der Angriffslane an der eigenen Basis.


\subsubsection{Truppen}

Truppen kosten relativ wenig, sind jedoch nicht kontrollierbar. Mit ihrem Spawn
suchen sie sich den kürzesten Weg zur gegenerischen Basis, um dort Schaden zu
verursachen. Erreicht eine Einheiten dieses Ziel, heißt sie \emph{erfolgreich.}


\subsubsection{Helden}

Helden kosten mehr als Truppen, sind jedoch in dem Sinne vollständig
kontrollierbar, dass der Spieler jederzeit bestimmen kann, wohin sich ein Held
bewegen soll. Den kürzesten Weg zu diesem Ziel sucht sich der Held daraufhin
selbst. Ist ein Ziel für eine Einheit nicht erreichbar, so wird der nähste
erreichbare Punkte als Ziel gewählt.

Jeder Held hat zusätzlich eine Fähigkeit, die den Spieler beim Angriff
unterstützt. Fähigkeiten sind entweder \emph{aktiv} -- der Spieler führt sie
explizit aus -- oder \emph{passiv} -- sie wird vom Helden selbst ausgeführt,
wenn er die Möglichkeit hat. Jede Fähigkeit hat eine Abklingzeit, die vergehen
muss, bevor sie erneut aktiv werden kann. Nach dem Spawn kann die Fähigkeit
direkt aktiv werden.

Es gibt unterschiedliche Helden, pro Spieler ist es jedoch nicht möglich mehr
als einen Helden jeder Art zur gleichen Zeit auf dem Spielfeld zu haben.
Stirbt ein Held, kann ein neuer Held dieser Art gekauft werden.
