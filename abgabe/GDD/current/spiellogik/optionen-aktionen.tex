\section{Optionen und Aktionen}

% Siehe auch: https://sopra.informatik.uni-freiburg.de/soprawiki/Game_Mechanic

% Dieser Abschnitt beinhaltet die Aktionen, die Spieler oder KI vornehmen
% können, um den Zustand des Spiels zu verändern (zB das Bauen von Einheiten
% oder das Abbauen von Ressourcen). Je klarer diese Aktionen formuliert sind,
% desto leichter fällt einem die Umsetzung der Aktionen bei der Programmierung
% des Spiels.
%
% Wichtig sind auch die Einstellungen, die der Spieler am Spiel vornehmen kann,
% um das Spielverhalten zu verändern (zB.  Schwierigkeitsgrad ändern).
%
% Das Ziel des Spiels sollte schließlich anhand der beschriebenen Aktionen
% erklärt werden.
%
% Die Auflistung der Optionen und Aktionen erfolgt tabellarisch und ist in Form
% und Inhalt an Use Cases (http://de.wikipedia.org/wiki/Use_case) angelehnt.

\StartId{A}
\begingroup
  \small
  \tabulinesep=1mm
\begin{longtabu}{X[0.6L]X[0.4L]X[L]X[L]X[L]}
  \rowfont{\normalsize}
  \caption{Mögliche Optionen und Aktionen\label{tab:optionen-aktionen}}\\
  \midrule[\heavyrulewidth]\rowfont{\itshape}
    ID/Name              &
    Akteure              &
    Ereignis"-fluss      &
    Anfangs"-bedingung   &
    Abschluss"-bedingung \\
  \midrule\endfirsthead

  \rowfont{\normalsize}
  \caption[]{Mögliche Optionen und Aktionen (fortges.)}\\
  \midrule[\heavyrulewidth]\rowfont{\itshape}
    ID/Name              &
    Akteure              &
    Ereignis"-fluss      &
    Anfangs"-bedingung   &
    Abschluss"-bedingung \\
  \midrule\endhead

  \multicolumn{5}{r}{\itshape fortges. auf der nächsten Seite}\\
  \endfoot

  \endlastfoot

  \defid[object-selection]{Objekt auswählen}
    & Spieler
    & Spieler klickt mit der Linken Maustaste auf einen Turm oder eine Einheit.
    & \todo[inline]{Keine Voraussetzung?}
    & Turm oder Einheit ist ausgewählt.
  \\\midrule

  \defid[cancel-object-selection]{Objektauswahl kündigen}
    & Spieler
    & Spieler klickt mit der Linken Maustaste auf einen Bereich des
      Spielfeldes, der keine andere Aktion auslöst.
    & Türme oder Einheiten sind ausgewählt.
    & Türme und Einheiten sind nicht mehr ausgewählt.
	\\\midrule

  \defid[move-figure]{Figur(en) durch Klick bewegen}
    & Spieler
    & \vspace*{-0.2cm}\begin{enumerate}[nosep,leftmargin=*]
        \item Spieler klickt mit der Rechten Maustaste auf einen Punkt auf der
          Angriffsstrecke.
        \item Die ausgewählten kontrollierbaren Einheiten bewegen sich auf dem
          kürzesten Weg auf den ausgewählten Punkt zu, Hindernisse werden
          rechtzeitig umlaufen.
      \end{enumerate}
    & Mindestens eine kontrollierbare Einheit ist ausgewählt.
    & Die Spielfiguren befinden sich am Zielpunkt, \textbf{\textls{oder}} die
      Spielfiguren befinden sich an dem erreichbaren Punkt, der möglichst
      nah am Zielpunkt liegt.
  \\\midrule

  \defid[hero-ability]{Heldenfähigkeit aktivieren}
    & Spieler
    & Der Spieler klickt mit der Linken Maustaste auf die Heldenfähigkeit.
    & Ein Held ist ausgewählt und seine Fähigkeit kann aktiviert werden.
    & Fähigkeit wird ausgeführt, siehe Tab.~\ref{tab:helden}.
  \\\midrule

  \defid[buy-attack]{Angriffseinheit kaufen}
    & Spieler
    & \vspace*{-0.2cm}\begin{enumerate}[nosep,leftmargin=*]
        \item Spieler klickt mit der Linken Maustaste auf eine Einheit in der
          Liste der Angriffseinheiten.
        \item Dem Spieler werden die Kosten der Einheit von seinem Gold
          abgezogen.
      \end{enumerate}
    & Der Spieler hat genügend Gold um die Einheit zu kaufen.
    & Beim Beginn der nächsten Welle spawnt eine Einheit mehr der ausgewählten
      Sorte.
  \\\midrule

  \defid[tower-attack]{Gebäude verteidigt}
    & Ge"-bäude
    & Das Gebäude führt seinen Angriff durch.
    & Einheit befindet sich in der Reichweite des Gebäudes und das Gebäude ist
      bereit zum Angriff.
    & Turm muss Angriffsintervall abwarten um den nächsten Angriff durchführen
      zu können; Einheiten werden getroffen (\refid{A:unit-hit}).
  \\\midrule

  \defid[unit-hit]{Angriffseinheit erhält Schaden}
    & Angriffs"-einheit
    & Die Angriffseinheit verliert Leben entsprechend der Verteidigungsstärke
      des Turms.
    & Einheit wurde von einem Turm angegriffen (\refid{A:tower-attack}).
    & Angriffseinheit hat weniger Lebenspunkte.
  \\\midrule

  \defid[die]{Angriffseinheit stirbt}
    & Angriffs"-einheit
    & \vspace*{-0.2cm}\begin{enumerate}[nosep,leftmargin=*]
        \item Einheit stirbt.
        \item Wenn die Einheit \emph{Trojaner} ist, wird
          \refid{A:spawn-children} ausgeführt.
      \end{enumerate}
    & Angriffseinheit hat 0 oder weniger Lebenspunkte.
    & Angriffseinheit nicht mehr sichtbar und nicht mehr verfügbar.
  \\\midrule

  \defid[spawn-children]{Trojaner spawnt Einheiten}
    & \emph{Tro"-janer}
    & An der Stelle an der die Eineit gestorben ist, werden \emph{Bugs} und
      \emph{Viren} gespawnt. Diese werden der gleichen Welle zugeordnet wie der
      Trojaner.
    & Trojaner ist gestorben.
    & \emph{Bugs} und \emph{Viren} sind gespawnt worden.
  \\\midrule

  \defid[damage-base]{Gegnerische Basis angreifen}
    & Angriffs"-einheit.
    & \vspace*{-0.2cm}\begin{enumerate}[nosep, leftmargin=*]
        \item Die Basis verliert Ladung entsprechend der Angriffsstärke der
          Einheit.
        \item Angriffseinheit wird gelöscht.
      \end{enumerate}
    & Angrffseinheit hat die gegnerische Basis erreicht.
    & Gegnerische Basis hat weniger Ladung, die Angriffseinheit ist nicht mehr
      verfügbar.
  \\\midrule

  \defid[base-die]{Basis stirbt}
    & Basis
    & Das Spiel ist beendet.
    & Basis hat eine Ladung von 0\,\% oder weniger.
    & Das Spiel ist beendet, der Spieler, dessen Basis mehr Ladung hat, hat
      gewonnen.
  \\\midrule

  \defid[select-building]{Verteidigungsgebäude auswählen}
    & Spieler
    & Spieler klickt mit der Linken Maustaste in der Liste der
      Verteidigungsgebäude auf ein Gebäude.
    & \todo[inline]{Keine Voraussetzung?}
    & Spieler befindet sich im Baumodus und kann eine Gebäudeplatzierung
      wählen (\refid{A:choose-position}).
  \\\midrule

  \defid[deselect-building]{Baumodus verlassen}
    & Spieler
    & Spieler klickt mit der Linken Maustaste auf das aktuell gewählte Gebäude
    & Spieler befindet sich im Baumodus.
    & Spieler befindet sich nicht mehr im Baumodus.
  \\\midrule

  \defid[choose-position]{Gebäudeplatzierung wählen}
    & Spieler
    & Der Spieler wählt mit der Maus einen gültigen Ort für ein Gebäude.
    & Der Spieler befindet sich im Baumodus.
    & Unter der Annahme, dass das Gebäude an der gewählten Position gebaut
      würde, muss für jede lebende Einheit ein Weg zu jeder Basis frei sein und
      ein Weg zwischen den Basen bestehen.
  \\\midrule

  \defid[put-defend]{Gebäude platzieren}
    & Spieler
    & Der Spieler klickt mit der linken Maustaste, das Gebäude wird platziert
      und die Kosten des Gebäudes vom eigenen Gold abgezogen.
    & Der Spieler hat eine gültige Gebäudeplatzierung gewählt
      (\refid{A:choose-position}) und genügend Gold um das Gebäude zu kaufen.
    & Das Gold wurde reduziert und das Gebäude platziert, das auf den
      nächstmöglichen Zeitpunkt seiner Aktivierung wartet.
      (\refid{A:activate-building}).
    \\\midrule

  \defid[activate-building]{Gebäude aktivieren}
    & Ge"-bäude
    & Das Gebäude aktiviert sich.
    & Das Gebäude ist entweder \emph{Schockfeld} \textbf{\textls{oder}} an der
      gewählten Position befinden sich keine Einheiten.
    & Das Gebäude ist aktiviert und kann feindliche Einheit angreifen
      (\refid{A:tower-attack}).
  \\\midrule

  \defid[tower-sell]{Gebäude verkaufen}
    & Spieler
    & Der Spieler erhält Gold zurück und das Gebäude verschwindet.
    & Ein eigenes Gebäude ist ausgewählt.
    & Der Spieler hat Gold zurückerhalten und das Gebäude ist weg.
  \\\midrule

  \defid[tower-improve]{Gebäude verbessern}
    & Spieler
    & \vspace*{-0.2cm}\begin{enumerate}[nosep, leftmargin=*]
        \item Spieler wählt die Verbesserung aus.
        \item Die Verbesserungskosten werden vom Gold des Spielers abgezogen.
        \item Der Turm wird verbessert.
      \end{enumerate}
    & Ein eigenes Gebäude ist ausgewählt und Spieler hat genügend Gold.
    & Der ausgewählte Turm wurde verbessert und das Gold abgezogen.
  \\\midrule

  \defid[tower-strategy]{Strategie wählen}
    & Spieler
    & Der Spieler klickt mit der Linken Maustaste auf die gewünschte Strategie.
    & Ein eingens Gebäude (nicht \emph{Kabel}) ist ausgewählt.
    & Das Gebäude handelt nun nach der gewählten Strategie.
  \\\midrule

  \defid[select-upgrade]{Upgrade auswählen}
    & Spieler
    & Der Spieler klickt mit der Linken Maustaste auf ein Upgrade, das
      angewendet wird.
    & Es sind genügend Erfahrungspunkte verfügbar.
    & Upgrade wurde angewendet.
  \\\midrule

  \defid[object-info]{Informationen anzeigen}
    & Spieler
    & Informationen über das Objekt werden eingeblendet.
    & Der Spieler hovert mit dem Mauszeiger über Gebäuden in der Liste der
      Verteidigungsgebäude, Einheiten in der Liste der Angriffseinheiten oder
      einem Upgrade.
    & Informationen werden eingeblendet.

  \\\bottomrule
\end{longtabu}
\endgroup
