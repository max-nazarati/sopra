\section{Spielstruktur}

% Dieser Abschnitt erklärt den Ablauf des Spiels. Das heißt, hier wird
% beschreiben, was geschieht, sobald der Spieler ein neues Spiel beginnt und
% wie sich das Spiel von dort aus entwickelt, bis es gewonnen oder verloren
% ist. Eine Beschreibung der unterschiedlichen Spielphasen ist hier essentiell.
%
% Eine mögliche Einteilung der Spielphasen von Schach ist zum Beispiel:
% Early-Game (Eröffnung), Mid-Game (strategische Positionen festigen),
% Late-Game (wenn nur noch wenige Figuren auf dem Brett sind).
%
% Eine weitere wichtige Information in diesem Abschnitt ist, welche Modi das
% Spiel hat (zum Beispiel Missionen und wie sie sich unterscheiden vs.
% Endlosmodus und wie dieser während des Spiels verändert wird).
%
% Außerdem soll die Dynamik des Spiels beschrieben werden (Beispiel: statisch,
% d.h. wenig Veränderungen an der Spielwelt und den Mechaniken vs. actionreich
% und dynamisch).
%
% Auch mögliche Taktiken und Strategien im Spiel können hier beschrieben
% werden.

Das Spiel ist dynamisch. Der Spieler und die KI verändern die Bebauung der
Lanes zur Verbesserung der Verteidigung und damit die Laufrouten der Truppen
während des Spiels durchgängig. Es gibt keine Ruhepausen, sobald das Spiel
einmal gestartet wurde. Die Angriffswellen laufen automatisch direkt
nacheinander ab, der Spieler hat darauf keinen Einfluss.

\subsection{Kaufoptionen}
\label{subsec:kaufoptionen}
    Alle Angriffseinheiten und Verteidigungsgebäude, deren Verbesserungen und
    die Upgrades können zu jedem Zeitpunkt während des Spiels gekauft werden, 
    solange der Spieler genügend Bitcoin hat. Unterschiede bestehen beim 
    Aktivierungs-/Spawnzeitpunkt.

  \begin{description}[noitemsep]
    \item[Angriffseinheiten] Helden spawnen direkt, Truppen beim
      Start der nächsten Welle.
    \item[Verteidigungsgebäude] werden aktiv, sobald sich an ihrem Platz keine
      Einheit befindet.
		\item[Verbesserungen] Die Veränderung tritt sofort in Kraft.
		\item[Upgrades] Die Veränderung tritt sofort in Kraft.
	\end{description}

\subsection{Spielablauf}

Beim Starten eines Spiels landet man in der Spielwelt. Der Spieler hat zu
Beginn fünfzig Bitcoin und eine halbe Minute Vorbereitungszeit, sprich seine
Gebäude aufzustellen und Angriffseinheiten zu kaufen.

Mit Ablauf der Vorbereitungszeit startet die erste Welle, alle bis dahin
gekauften Angriffseinheiten spawnen in kurzem Abstand an ihrer Basis und
bewegen sich auf dem kürzesten Weg zur gegnerischen Basis. Ihre Reihenfolge
entspricht dabei der Kaufreihenfolge. Zusätzlich können ab jetzt die gekauften
Helden bewegt werden.

Sobald das Spiel gestartet wird verdient der Spieler pro Sekunde einen Bitcoin,
die direkt investiert werden können (siehe
Abschnitt~\ref{subsec:kaufoptionen}).

Der Spieler nutzt nun entweder seine Helden, um seine Einheiten im Angriff zu
unterstützen, verbessert seine Verteidigung durch das Bauen weiterer Türme oder
kauft neue Angriffseinheiten, die mit der nächsten Welle spawnen.

Das Spiel geht solange, bis die Ladung einer Basis auf 0\,\% oder weniger
gesunken ist. Der Spieler, dessen Basis mehr Ladung hat, hat dann gewonnen. Dem
Spieler wird nun das Ergebnis angezeigt und er kann ins Hauptmenü zurückkehren.

\subsection{Verhalten der KI}
Anhand integrierter Metriken entscheidet sich die KI mehr Bitcoins in Offensive, Defensive oder Upgrades aus der Spielfeldmitte zu investieren - sie
handelt bei der Auswahl in welche Einheiten sie investieren sollte sehr prophylaktisch: 
In Abhängigkeit von der Komposition der Angriffseinheiten des Gegner und der eigenen
Verteidigung werden entsprechende Verteidigungseinheiten gebaut.
