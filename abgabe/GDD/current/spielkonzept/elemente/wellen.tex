\subsection{Wellen}

\subsubsection{Terminologie}

Eine Welle besteht aus zwei \emph{Wellenteilen.} Jedes Wellenteil enthält
Truppen eines Spielers.  Sind alle Einheiten eines Wellenteils entweder
gestorben oder erfolgreich in die gegnerische Basis eingedrungen, so gilt
dieser Teil als „vom Gegner \emph{besiegt}“.

Zwei Wellenteile, nämlich das des Spielers und das des Gegenspielers, bilden
zusammen eine \emph{Welle.} Eine gestartete Wellen, bei der mindestens ein Teil
noch nicht besiegt wurde, heißt \emph{aktiv}.
Solange von einer aktiven Welle noch kein Teil besiegt wurde, nennen wir sie die \emph{aktuelle} Welle.


\subsubsection{Wellenstart}

Die erste Welle startet dreißig Sekunden nach Beginn des Spiels. So hat man also eine halbe Minute um sich vorzubereiten.
Für alle weiteren Wellen gilt folgende Regelung: Die nächste Welle startet in dem Moment, in dem von der aktuellen Welle, ein Teil besiegt wird.


\subsubsection{Zusammensetzung}

Ein Kauf einer Truppe ist vielmehr der Kauf eines Spawners, der beim Beginn
jeder zukünftigen Welle eine Einheit dieser Art spawnt. Im Verlauf des Spiels
werden die Wellen somit immer größer. Alle Einheiten die mit dem Beginn einer
Welle gespawnt werden, sind dem entsprechenden Wellenteil zugeordnet.

Helden sind keiner Welle zugeordnet. Beim Kauf einer solchen Einheit wird sie
direkt an der eigenen Basis gespawnt.
